% DRC: Digital Room Correction
% User manual
%
% Copyright (C) 2002-2012 Denis Sbragion
%
% This program is free software; you can redistribute it and/or modify
% it under the terms of the GNU General Public License as published by
% the Free Software Foundation; either version 2 of the License, or
% (at your option) any later version.
%
% This program is distributed in the hope that it will be useful,
% but WITHOUT ANY WARRANTY; without even the implied warranty of
% MERCHANTABILITY or FITNESS FOR A PARTICULAR PURPOSE.  See the
% GNU General Public License for more details.
%
% You should have received a copy of the GNU General Public License
% along with this program; if not, write to the Free Software
% Foundation, Inc., 675 Mass Ave, Cambridge, MA 02139, USA.
%
% You can contact the author on Internet at the following address:
%
%		d.sbragion@infotecna.it
%
% This program uses the FFT  routines from Takuya  Ooura and the GNU
% Scientific Library (GSL)  FFT routines. Many  thanks to Takuya Ooura and
% the GSL developers for these efficient routines.

% Definizione stile documento
\documentclass[a4paper,titlepage]{article}
\pagestyle{headings}

% Pacchetti aggiuntivi
\usepackage{url}
\usepackage{graphicx}
\usepackage{tocloft}
\usepackage[pdftitle={DRC: Digital Room Correction},pdfauthor={Denis Sbragion},
a4paper,pdfstartview=FitH,plainpages=false,pdfpagelabels,
colorlinks=true,bookmarks=true,bookmarksnumbered,linktocpage]{hyperref}

% Incrementa lo spazio per la numerazione nella tavola dei contenuti
\addtolength{\cftsubsubsecnumwidth}{0.5em}

% Dimensione figure
\newcommand{\basepctwidth}{0.9}
\newcommand{\pctwidth}{1.0}

% Definizione titolo
\title{DRC: Digital Room Correction}

\author{Denis Sbragion}

\date{
2012-10-01
\linebreak\linebreak
\small{Copyright \copyright\ 2002-2012 Denis Sbragion}
\linebreak\linebreak
\small{Version 3.2.1}
}

% Inizio documento
\begin{document}

% Imposta il page numbering a roman per la pagina del titolo
\pagenumbering{roman}

% Pagina con il titolo
\begin{titlepage}

\maketitle\thispagestyle{empty}

\end{titlepage}

% Reimposta il page numbering a arabic per le pagine rimanenti
\pagenumbering{arabic}

% Pagina con la licenza GPL

\thispagestyle{empty}
{\sf

\noindent This program is free software;  you can redistribute it and/or
modify it under the terms of the GNU General Public License as published
by the Free Software Foundation; either version 2 of the License, or (at
your option) any later version.
\medskip

\noindent  This  program  is distributed  in  the hope  that it  will be
useful, but WITHOUT ANY  WARRANTY; without even  the implied warranty of
MERCHANTABILITY or FITNESS FOR A PARTICULAR PURPOSE. See the GNU General
Public License for more details.
\medskip

\noindent You  should  have received  a copy of  the GNU  General Public
License along with this program; if not, write to the Free Software
Foundation, Inc., 675 Mass Ave, Cambridge, MA 02139, USA.
\medskip

\noindent  You can  contact  the  author on  Internet  at the  following
address:

\begin{quote}
\url{d.sbragion@infotecna.it}
\end{quote}

\noindent This program uses the  FFT routines from Takuya Ooura and
the GNU Scientific  Library  (GSL) FFT  routines. Many  thanks to Takuya
Ooura and the GSL developers for these efficient routines.

}

% Salto pagina tavola dei contenuti
\clearpage

%Tavola contenuti
\tableofcontents

% Salto pagina lista figure
\clearpage

% Lista figure
\listoffigures

% Salto pagina inizio manuale
\clearpage

% Introduzione
\section{Introduction}

DRC uses a lot of  signal, linear  system and DSP theory  to achieve its
results.  In this user manual some  knowledge about  those
arguments is assumed. I'm planning to write a more extensive manual with
the basics needed to understand what  DRC does, or at least is trying to
do, but unfortunately I have just  little spare time to dedicate to this
project,   so I  will   concentrate   just  on  improving   the  program
performances.   Of  course  volunteers,   suggestion,   patches,  better
documentation,     pointers  and  references   on the  subject  are  all
appreciated.

For a basic introductory guide to DSP theory and practice you might look
at:

\begin{quote}
\url{http://www.dspguide.com/}
\end{quote}
For a basic introduction to DSP applied to audio you might read the good
book available at:

\begin{quote}
\sloppy\raggedright
\url{http://profs.sci.univr.it/~rocchess/htmls/corsi/SoundProcessing/SoundProcessingBook/}
\end{quote}
To better understand what DRC is trying to do you might look  at:

\begin{quote}
\url{http://en.wikipedia.org/wiki/Digital_Room_Correction}
\end{quote}
This clear and concise Wikipedia  article contains all the basics needed
to understand digital  room correction  in general.  Another interesting
page is available at:

\begin{quote}
\url{http://www.ludd.luth.se/~torger/filter.html}
\end{quote}
On this web  page you'll find  some good  explanations  about the Nwfiir
Audio Tools suite, which  was a project, now  discontinued, similar to
DRC but implemented using warped FIR filters instead of the usual standard
FIR filters.

Compared with the Nwfiir Audio Tools suite DRC does only the job carried
out by the wfird  program,  generating just  the FIR filters for
digital room correction.

In order to measure the  room impulse response
and to perform real time or offline convolution (i.e. the correction) of
the digital signal,  you have to use  some external  programs, like, for
example, BruteFIR (see section \ref{BruteFIR}).

A good DRC step by step guide has been written by ``Jones Rush'', and it
is available at the following URL:

\label{JonesRushGuide}

\begin{quote}
\url{http://www.duffroomcorrection.com/images/d/de/DRC_Guide_v1.0.pdf}
\end{quote}
``Jones Rush'' spent quite a lot of time learning the complete procedure
needed to set up a full digital room  correction system and also spent a
lot of time writing the guide from the beginner's point of view, so this
is a really good starting point for everyone who has never played before
with this sort of things.  The guide is now a bit  outdated, but despite
this it is still a valid  reference for the whole  procedure of
DRC  filters creation.  The  main  difference  is  in the name  of the  output  file
generated  by  the latest   sample  configuration  files,  which  now is
``rps.pcm'' instead of ``dxf.pcm''.

Ed Wildgoose is trying to create a collaborative documentation effort at:

\begin{quote}
\url{http://www.duffroomcorrection.com/}
\end{quote}
Please, take  the time  to improve  the available  documentation  and to
share your experience  participating to those  nice Wiki pages. It could
be really useful for other DRC users.

% Riferimenti
\section{Getting the latest version}

The official DRC web site is available at the following address:

\begin{quote}
\url{http://drc-fir.sourceforge.net/}
\end{quote}
On the web site you will  always find news and  up to date informations,
the full documentation for the latest  version, informations about where to
download it and  many other DRC related  informations.  New DRC releases
are announced using the Freshmeat announcement and tracking service. The
Freshmeat DRC page is available at the following address:

\begin{quote}
\url{http://freshmeat.net/projects/drc/}
\end{quote}

% Nuove funzionalit�
\section{What's new in version 3.2.1}

Minor accuracy improvements and minor changes to the default
configuration templates have been introduced. Configuration file handling 
now is performed using the minIni library instead of the parsecfg library. 
A minor bug in the axis labeling of the Morlet cycle/octave scalograms 
has been corrected. The plot generation scripts have been reworked for better
readability, better axis labeling and compatibility with the latest version
of Octave.

\subsection{Compatibility with previous versions}

Configuration files for version 3.2.0 are compatible with the
the current version, but it is suggested to change PLStartFreq and
PLEndFreq to match the new default values, i.e. 100 and 10000
respectively.

% Modifiche precedenti
\subsection{History: what was new in previous versions}

\subsubsection{Version 3.2.0}

A new computation of the RMS value based on a logarithmic frequency
weighting of the magnitude response has been introduced. This new method
allow for a better handling of the filter magnitude response peak
limiting. A bug in the extimation of the band limited RMS value has been
also corrected. Some other procedures have been refined in order to
provide an improved accuracy. The glsweep and lsconv tools have been
reworked to provide correct reference levels thus allowing for SPL
calibrated measurements.

\subsubsection{Version 3.1.1}

The licensing of some file has been corrected. Minor corrections
and accuracy improvements.

\subsubsection{Version 3.1.0}

The Octave scripts have been reworked to make them compatible with the
latest version of Octave and improved to provide some autoscaling
features and exports to different image formats. The microphone
compensation stage has been moved to the beginning of the correction
procedure so that any following stage works using the compensated
impulse response as a reference. A new parameter adding a configurable
delay to the minimum phase version of the correction filter has been
introduced. Many other minor bugs have been fixed.

\subsubsection{Version 3.0.1}

The Octave scripts have been reworked to make them compatible with
version 3 of Octave. A new renormalization procedure, providing a
reasonable extimation of clipping levels, has been added. Many minor
bugs have been fixed.

\subsubsection{Version 3.0.0}

A new method for the computation of an optimized psychoacoustic target
response, based on the spectral envelope of the corrected impulse
response, has been introduced.

\subsubsection{Version 2.7.0}

A new method for the computation of the excess phase component inverse,
based on a simple time reversal, has been introduced. The sample
configuration files have been rewritten to take advantage of the new
inversion procedure. Sample configuration files for 48 kHz, 88.2 kHz, 96
kHz sample rates have been added. The homomorphic deconvolution
procedure has been improved to avoid any numerical instability. A new
Piecewise Cubic Hermite Interpolating Polynomial (PCHIP) interpolation
method, providing monotonic behaviour, has been introduced in the target
response computation. All the interpolation and approximation procedures
have been rewritten from scratch to provide better performances and
accuracy.

\subsubsection{Version 2.6.2}

A new  command  line  parameters   replacement  functionality   has been
introduced. The dip and  peak limiting procedures  have been improved in
order to avoid  numerical  instabilities.  A new  wavelet based  analysis
graph  has  been   added  to the   sample  results.   Many   performance
improvements  have been  introduced.  A new  optional parameter  used to
define the base directory for all files has been added.

\subsubsection{Version 2.6.1}

Minor   corrections    and   improvements   have  been  applied   to the
documentation and to the pre-echo  truncation inversion procedure. A new
target transfer function definition procedure based on Uniform B Splines
has been  introduced.  The  development  environment  has been  moved to
Code::Blocks and GCC/MinGW.

\subsubsection{Version 2.6.0}

A new prefiltering curve  based on the bilinear  transformation has been
introduced. An improved  windowing of the minimum  phase filters used to
apply the target frequency response  and the microphone compensation has
been   implemented.  A  missing   normalization  of  the  minimum  phase
correction filter  has been added. A  new logarithmic  interpolation has
been  added  to the  target  transfer   function  computation.   The new
interpolation method  simplifies the  definition of the  target transfer
functions. Small  improvements  to the  documentation and  to the Octave
scripts used to  generate the graphs  have been applied.  A new improved
version of the measurejack script has been included in the package. Some
new sample  configuration  files,  including one  approximating  the ERB
psychoacoustic scale, have been added.

\subsubsection{Version 2.5.1}

Small improvements to the  documentation and to  the Octave scripts used
to generate the graphs.  The sliding lowpass  prefiltering procedure has
been rewritten to make it a bit more  accurate and to make the code more
readable. Few other minor bugs have been fixed.

\subsubsection{Version 2.5.0}

With version  2.5.0  a  general  overhauling  of the  filter  generation
procedure has  been performed.  Some steps  (peak limiting  for example)
have been moved to  a different stage  of the procedure,  and new stages
have been added.

A new  ringing  truncation  stage  has been  added to  remove  excessive
ringing caused sometimes  by the pre-echo truncation  procedure. Now the
filter impulse  response is  enclosed in a  sort of  psychoacoustic jail
that prevent, or at least  reduces a lot, any  artifact that could arise
as a side effect of the  filter generation  procedure. With this changes
DRC becomes somewhat ``self tuning''  and now it is able to adapt itself
to the input  impulse response,  at least  to some extent,  providing as
much correction as possible without generating excessive artifacts.

The postfiltering stage, where the  target transfer function is defined,
has been split to provide a separate  stage for microphone compensation.
This allows for a greater flexibility  defining both the target transfer
function and the microphone compensation,  and provides as a side effect
correct test convolutions even when microphone compensation is in place.
With the previous versions  the test convolution  was improperly altered
by the  microphone   compensation,   because  both the  target  transfer
function and  the  microphone  compensation were  generated  and applied
using the same filter.

Many other procedures  have been  refined. For example  the peak and dip
limiting procedures now ensure  continuity up to the first derivative of
the magnitude response on the points where the magnitude limiting starts
its effect. This further reduces the ringing caused by abrupt changes in
the  magnitude   response.

Finally many other minor bugs have  been corrected and the documentation
has been  improved,  switching to  \LaTeX\ for  document  generation and
formatting.

\subsubsection{Version 2.4.2}

Version  2.4.2 added  a better  handling  of underflow  problems  during
homomorphic  deconvolution.  Some little  speed improvements  have been
also achieved. Added  search and output of peak  value and peak position
into lsconv.

\subsubsection{Version 2.4.1}

Version  2.4.1  added  some  tools for  accurate  time  aligned  impulse
response   measurements.   This  make  it  possible  to  compensate  for
interchannel misalignments, at least  up to a limited extent. Some minor
bugs have been also corrected.

\subsubsection{Version 2.4.0}

In version  2.4.0  the  Takuya  Ooura  and GNU  Scientific  Library  FFT
routines have been included in the  program. These routines are about 10
times  faster than  the  previous  routines,  providing  about  the same
accuracy.  Furthermore  some  checks have  been added  to the  sharpness
parameters to avoid program crashes when these parameters are missing.

The FFT routines described above are available at:

\begin{quote}
\url{http://www.gnu.org/software/gsl/}\newline
\url{http://www.kurims.kyoto-u.ac.jp/~ooura/fft.html}
\end{quote}

\subsubsection{Version 2.3.2}

In version 2.3.2 a new sharpness  factor parameter has been added to the
sliding low  pass  prefiltering  procedure.  This  parameter  provides a
control between filtering sharpness and spectral spreading in the filter
transition  region.  A new  option to  read and write  double  precision
floating points files has been added. Some checks to warn when the input
signal is too short to provide accurate results has been added.

\subsubsection{Version 2.3.1}

In version  2.3.1  some  minor  corrections  to  the  program  have been
performed and the  documentation has been  restructured. A new option to
automatically  count  the number  of lines  in the  target  function and
microphone  compensation files  has been  added. A new  optimized sample
configuration file has been added.

\subsubsection{Version 2.3.0}

Version  2.3.0   adds two   parameters  to  control  the  gain  limiting
procedures. These parameters control  a sort of ``soft clipping'' of the
frequency  response,   avoiding  ringing on  abrupt  truncations  of the
frequency response. A new parameter to select the magnitude type, either
linear or  expressed in dB,  of the target  frequency  response has been
added. The optional  capability to  perform microphone  compensation has
also been added. The license has been switched to the GNU GPL.

\subsubsection{Version 2.2.0}

Version  2.2.0  added a  sliding  low  pass  procedure  to the  pre-echo
truncation inversion  procedure. This  pre-echo truncation  procedure is
much more  similar  to  the  pre-echo  sensitivity  of  our hear  and so
slightly better results  are achieved. Furthermore  the sliding low pass
prefiltering procedure  has been completely  rewritten to provide better
accuracy, especially with  the short window lengths  needed for pre-echo
truncation.

\subsubsection{Version 2.1.0}

Version 2.1.0 added two  new parameters that allow  for the windowing of
everything  coming  more than  few samples  before  the impulse  center.
Usually before  the main spike  there's only  noise and  spuriae. I have
found that in certain  situations  this small noise may  lead to audible
errors in the  correction,  so  windowing it out  in order  to clean the
impulse response is a good practice.

\subsubsection{Version 2.0.0}
\label{Version-2.0.0}

Version 2.0.0 added many new features  that provides much better control
on pre-echo  artifacts problems.  The most  important  change is the new
pre-echo truncation  inversion  procedure. Loosely  derived from Kirkeby
fast deconvolution  this  new procedure  truncates  any  pre-echo on the
excess phase part inversion. This  leads to something like minimum phase
inversion on frequency  ranges where a complete  inversion would lead to
pre-echo artifacts. This  critical frequency  ranges are usually no more
than 5 or 6 and no  wider than about  1/12 of octave  for a typical room
impulse response. Reducing the  correction to minimum phase on so narrow
bands has little or no  subjective effect on the  correction quality and
allows for  the  correction of  much  longer windows,  with  much better
overall results.

Avoiding  pre-echo  artifacts also  provides the  ability to  create low
input-output delay filters. The resulting delay is often low enough (few
ms) to allow the use of these filters  in home theater applications. For
situations where even few ms aren't  adequate there's now also an option
to generate  zero delay  minimum  phase filters.  Minimum  phase filters
provides correction of the amplitude response and just the minimum phase
part of the phase response.

In order to avoid pre-echo  artifacts there are  also many other aspects
that should be taken into account. For a better explanation of the whole
procedure and  the  selection method  for the  DRC parameters  needed to
achieve this result look at the section \ref{AvoidPreEchoArtifacts}.

Version 2.0.0 adds  also many other  improvements,  including the single
side version of the  prefiltering  procedures and fixing  for many minor
bugs that where still laying around.

A test convolution stage is now also available. This convolves the input
impulse response with the  generated filter to  get the impulse response
after  correction.  The  impulse  response  obtained  by this  method is
usually really  reliable. As  long as the  measurement  microphone isn't
moved I have been able to verify the computed impulse response with less
than  $0.5$  dB errors,   which is   impressive  considering  the  cheap
measurement   set I  use.  In my  situation  may  be that  the  computed
corrected impulse response is even  more accurate than the measured one,
because of noise problems  being doubled by my  cheap measurement set in
the second measure.

\subsubsection{Version 1.3.0}

Version 1.3.0 provided some new features with respect to version
1.2.1:

\begin{itemize}

\item More flexible prefiltering curve parameters

\item New time varying sliding lowpass prefiltering stage

\item  New  minimum   phase  or  homomorphic    renormalization   of the
prefiltered excess phase component

\item Homomorphic  deconvolution based on the  Hilbert transform instead
of the cepstrum method

\item Slightly improved documentation

\item Many minor bugs fixed

\end{itemize}

\section{Program description and operation}

DRC is a  program  used to  generate  correction   filters for  acoustic
compensation of HiFi and  audio systems in general,  including listening
room compensation. DRC generates just  the FIR correction filters, which
must be used with a real time or  offline convolver to provide real time
or offline  correction. DRC  doesn't provide  convolution  features, and
provides  only some  simplified,  although  really  accurate,  measuring
tools. So in order to use DRC you need:

\begin{enumerate}

\item At least 1 second  of the impulse response  of your room and audio
system at the listening  position, separated for  each channel, which means
usually  just  left and  right  for a  basic  HiFi system.  The  impulse
response should be provided in raw  format (flat file with just samples,
no headers or  additional information  whatsoever),  either in signed 16
bit format or using 32/64 bit IEEE  floating point samples. From version
2.4.1 DRC includes some  command line tools to  do accurate time aligned
impulse                response        measurement,      see     section
\ref{ImpulseResponseMeasurement}      for further   details. Many  other
systems, either  commercial  or free, are  available on  the Internet to
carry out this task. Take a look at:

\begin{itemize}
 \sloppy\raggedright
 \item Room Eq Wizard \url{http://www.hometheatershack.com/roomeq/}
 \item rec\_imp: \url{http://www.duffroomcorrection.com/wiki/Simple_Automated_IR_Measuring_Tool}
 \item ETF: \url{http://www.etfacoustic.com/}
 \item Sample Champion: \url{http://www.purebits.com/}
 \item Aurora plugins: \url{http://www.aurora-plugins.com/}
 \item MLSSA: \url{http://www.mlssa.com/}
 \item CLIO: \url{http://www.audiomatica.it/}
 \item Audua Speaker Workshop: \url{http://www.speakerworkshop.com/}
 \item The MLS system provided with the Nwfiir Audio Tools suite
 \item LAUD
 \item Techron TEF
\end{itemize}
Since some time there are also some integrated packages provinding both
the measurements functionality and the correction, based on DRC. Some
examples:

\begin{itemize}
 \sloppy\raggedright
 \item DRCDesigner: \url{http://www.alanjordan.org/DRCDesigner/DrcDesignerHelp.html}
 \item Align: \url{http://www.ohl.to/about-audio/audio-softwares/align/}
 \item The Final Cut: \url{http://www.ohl.to/about-audio/audio-softwares/the-final-cut/}
 \item Inguz Audio: \url{http://inguzaudio.com/}
\end{itemize}
This packages may be really useful, expecially for unexperienced peoples.

Many information  and free  programs useful  for measuring  and handling
impulse responses are available at the NoiseVault web site:

\begin{quote}
\url{http://www.noisevault.com/}
\end{quote}
Of course a good instrumentation  microphone and preamplifier are needed
to get accurate  measurements of your  listening room  response. The ETF
web sites has  a link to a  cheap but still  quite good  instrumentation
microphone, which comes with an individual calibration file:

\begin{quote}
\url{http://www.ibf-akustik.de/}
\end{quote}

It is built around the Panasonic electrect capsules (WM-60A and WM-61A)
which can be used also to build a DIY microphone. Of course you won't
get the same quality of a professional instrumentation microphone, but
it is enough to get good results. Some other good and inexpensive solutions
are the Behringer ECM8000 measurement microphone (see
\url{http://www.behringer.com} for details) and the Dayton Audio EMM-6:

\begin{quote}
\url{http://www.daytonaudio.com/index.php/emm-6-electret-measurement-microphone.html}
\end{quote}
which provides also individual calibration files.

\item A real time convolver able to  deal with FIR filters with at least
4000 and up  to more  than 32000  taps, like  BruteFIR,  Foobar2000, the
ActiveX Convolver Plugin  and others. References  for these programs can
be found at the following links:

\label{BruteFIR}
\begin{itemize}
\sloppy\raggedright
 \item BruteFIR: \url{http://www.ludd.luth.se/~torger/brutefir.html}
 \item ActiveX Convolver Plugin: \url{http://convolver.sourceforge.net}
 \item Foobar2000 with convolver plugin: \url{http://www.foobar2000.org/}
 \item Jconv: \url{http://www.kokkinizita.net/}
 \item RealReverb WinAmp plugin, optionally along with the LineIn plugin:
  \begin{quote}
   \url{http://www-personal.umich.edu/~mressl/}\newline
	 \url{http://www.winamp.com/}\newline
   \url{http://home.hccnet.nl/th.v.d.gronde/}
  \end{quote}
 \item SIR Reverb plugin: \url{http://www.knufinke.de/sir/index_en.html}
 \item Aurora plugins: \url{http://www.aurora-plugins.com/}
 \item CATT FIReverb suite: \url{http://www.catt.se/the_suite.htm}
 \item Convo Boy: \url{http://www.kvraudio.com/get/1469.html}
 \item Voxengo Pristine Space: \url{http://www.voxengo.com/product/pspace/}

\end{itemize}
Furthermore   a  ready  to use  Linux   distribution  suited  for  audio
applications is available at Planet CCRMA:

\begin{quote}
\url{http://ccrma-www.stanford.edu/planetccrma/software/}
\end{quote}
This distribution  already contains  most of what is  needed to create a
real time convolution engine suited for digital room correction.

On the DRC  Wiki  pages  created by  Ed  Wildgoose  there's a  document,
created by Uli Brueggemann,  on how to create a  Linux mini distribution
suited to run BruteFIR out of a USB memory stick. Take a look at:

\begin{quote}
\url{http://www.duffroomcorrection.com/}
\end{quote}
and search for ``BruteFIR on a USB memory stick''. Further informations about
this option are available on the Acourate web site:

\begin{quote}
\url{http://www.acourate.com/}
\end{quote}

\item Hardware needed to run all the programs. I'm actually using a
``Shoe Box'' PC manufactured by ITOX (\url{http://www.itox.com}) along
with a TerraTec EWX 24/96 sound card. This ``Shoe Box'' PC is running
Linux (RedHat 7.3), ALSA (see \url{http://www.alsa-project.org}) and
BruteFIR to provide real time correction from the optical S/PDIF output
of a consumer CD player. The PC configuration is really simple (Intel
Celeron CPU running at 800 MHz, 64 Mb of RAM, old 1.6 Gb Hard Disk) but
it is more than adequate for real time correction of two channels
running at 44.1 kHz. With this configuration BruteFIR uses just about
15\% of the CPU power.

Of course to test the program you can also apply the correction off-line
on audio files ripped from ordinary audio CDs, burning the corrected
files on some fresh CDR and listening to them using a standard CD
player. This avoids the need of any dedicated hardware and lets you test
DRC on your favourite CD player.

Since few years there are many good, silent and compact PCs designed for
multimedia usage which could be used to build a complete noiseless real
time convolver with little effort. For some examples take a look at:

\begin{quote}
\url{http://www.cappuccinopc.com/}\newline
\url{http://www.stealthcomputer.com/}\newline
\url{http://www.tranquilpc.co.uk/}\newline
\url{http://www.mini-itx.com/}\newline
\end{quote}

\end{enumerate}

\subsection{Filter generation procedure}

The creation of a  correction filter  for room acoustic  compensation is
quite a  challenging  task.  A typical   acoustic  environment  is a non
minimum phase system, so in theory  it cannot be inverted to get perfect
compensation. Furthermore  a typical HiFi system  in a typical listening
room isn't either a single linear  system, but it is instead a different
linear system for every different listening position available.

Trying to  get an  almost  perfect  compensation  for  a given  position
usually leads to unacceptable  results for positions  which are even few
millimeters  apart  from the  corrected  position.  The  generation of a
filter that  provides good  compensation of  magnitude and  phase of the
frequency response of the direct sound, good control of the magnitude of
the  frequency   response   of the   stationary  field  and an acceptable
sensitivity on the listening position, requires many steps. Here it is a
brief summary of what DRC does:

\begin{enumerate}

\item Initial windowing and normalization of the input impulse response.

\item Optional microphone compensation.

\item Initial dip limiting to prevent numerical instabilities during
homomorphic deconvolution.

\item  Decomposition  into  minimum phase  and excess  phase  components
using homomorphic deconvolution.

\item  Prefiltering  of  the  minimum  phase  component  with  frequency
dependent windowing.

\item Frequency response dip limiting  of the minimum phase component to
prevent numerical instabilities during the inversion step.

\item   Prefiltering  of the  excess  phase   component  with  frequency
dependent windowing.

\item Normalization  and convolution  of the preprocessed  minimum phase
and excess phase components (optional starting from version 2.0.0).

\item Impulse response inversion through least square techniques or fast
deconvolution.

\item Optional computation of a psychoacoustic target response based on the
magnitude response envelope of the corrected impulse response.

\item   Frequency   response  peak   limiting  to  prevent  speaker  and
amplification overload.

\item Ringing  truncation with frequency  dependent  windowing to remove
any unwanted  excessive ringing  caused by  the inversion  stage and the
peak limiting stage.

\item Postfiltering  to remove  uncorrectable (subsonic  and ultrasonic)
bands and to provide the final target frequency response.

\item Optional generation  of a minimum phase  version of the correction
filter.

\item Final optional test convolution  of the correction filter with the
input impulse response.

\end{enumerate}
Almost each of these steps have configurable parameters and the optional
capability to output intermediate results.

Of course I'm  not sure at  all that this  is the best  procedure to get
optimal correction filters. There is a lot of psychoacoustic involved in
the generation of room acoustic  correction filters, so probably the use
of a more  psychoacoustic   oriented  procedure would  give  even better
results. Any suggestion with respect to this is appreciated.

Within my HiFi system the application of the correction provides huge
improvements. By the way my system no longer can be considered a normal
HiFi system. It is actually much closer to a studio monitoring system
placed in a heavily damped room. Furthermore there is also an high
performance subwoofer, extending the response down into the infrasonic
range, and everything has been tuned to provide the best results with
the correction in place. For some example of the results achieved see
appendix \ref{SampleResults}.

\subsection{Frequency dependent windowing}
\label{FrequencyDependentWindowing}

The frequency dependent  windowing is one of the  most common operations
within DRC. This type of windowing follow up directly from the fact that
within a  room the  sensitivity  of the  room transfer  function  to the
listening position is roughly dependent on the wavelength involved. This
of course implies that the listening position sensitivity increase quite
quickly with frequency.

This dependence has the side effect  that the room correction need to be
reduced as the frequency increase,  or, seen from the other side, as the
wavelength  decrease. For  this reason DRC  tries to apply  a correction
that is roughly proportional  to the wavelength  involved. This approach
has also some psychoacoustic  implications, because  our auditory system
is conceived to take into account the  same exact room behaviour, and so
its own behaviour follow somewhat similar rules.

Within DRC the frequency dependent windowing is implemented with two
different kind of procedures: band windowing and sliding lowpass linear
time variant filtering. The first procedure simply filters the input
signals into logarithmically spaced adjacent bands and applies different
windows to them, then summing the resulting signals together to get the
output windowed impulse response. The second procedure uses a time
varying lowpass filter, with a cut-off frequency that decreases with the
window length. The results are pretty similar, but usually the sliding
lowpass procedure is preferred because it is less prone to numerical
errors and allows for a bit more of flexibility.

Both  procedures  follow  the same  basic  rules in order to  define the  type of
windowing that gets  applied to the  input signal. The  basic parameters
are the lower window, i.e. the window  applied at the lower end of the
frequency range involved,  the upper window, i.e.  the window applied at
the upper end of  the frequency  range, and the window  exponent, i.e.
the exponent used to connect the lower window to the upper window with a
parametric function that goes as the inverse of the frequency. For
a   description   of  the    parametric   function   used  see   section
\ref{MPWindowExponent}.

\begin{figure}
\begin{center}
\includegraphics[width=\basepctwidth\textwidth,keepaspectratio]{figures/DBP-Linear}
\caption{\label{PrefilterLinear}  Frequency  dependent windowing for the
normal.drc sample  settings on the  time-frequency  plane. Linear scale.
The X axis is time in  milliseconds, the Y axis  is frequency in Hz. The
part that gets corrected is the one below the windowing curves.}
\end{center}
\end{figure}

For example figure \ref{PrefilterLinear} show the typical set of
prefiltering curves applied to the input impulse response by the
normal.drc sample settings (see section \ref{SampleConfigurationFiles}).
For this sample settings file the default setting for the window
exponent, which is the WE parameter in the figure, is 1.0. The part of
the input impulse response that is preserved, and so also corrected, is
the one below the curves. The remaining part of the time-frequency plane
is simply windowed out and ingored in the correction.

Looking at this figure it becomes pretty clear that only a tiny fraction
of the time-frequency  plane gets  corrected by DRC.  This tiny fraction
pretty much defines the physical limits where digital room correction is
applicable. Above this limit the  listening position sensitivity usually
becomes so  high that  even a small  displacement  of the  head from the
optimal   listening  position   causes  unacceptable   results  with the
appearance of strong audible artifacts.

\begin{figure}
\begin{center}
\includegraphics[width=\basepctwidth\textwidth,keepaspectratio]{figures/DBP-SemiLogY}
\caption{\label{PrefilterSemilogY} Frequency dependent windowing for the
normal.drc  sample  settings on the  time-frequency  plane.  Logarithmic
frequency  scale. The  X axis  is time  in milliseconds,  the  Y axis is
frequency in Hz.}
\end{center}
\end{figure}

By the way it should be  also taken into account  that our ear perceives
this time-frequency plane  on a logarithmic  frequency scale. Looking at
the  same  graph  on  a  logarithmic    frequency  scale,  as  in  figure
\ref{PrefilterSemilogY},  it becomes clear that  from our auditory system
point of view a  much bigger fraction  of the time-frequency  plane gets
corrected.  It becomes  also clear  that above  1-2 kHz only  the direct
sound gets corrected and that above  that range room correction actually
reduces to just minimalistic speaker correction.

\begin{figure}
\begin{center}
\includegraphics[width=\basepctwidth\textwidth,keepaspectratio]{figures/DBP-SemiLogX}
\caption{\label{PrefilterSemilogX} Frequency dependent windowing for the
normal.drc sample settings on the time-frequency plane. Logarithmic time
and linear frequency  scales. The X axis is time  in milliseconds, the Y
axis is frequency in Hz.}
\end{center}
\end{figure}

\begin{figure}
\begin{center}
\includegraphics[width=\basepctwidth\textwidth,keepaspectratio]{figures/DBP-LogLog}
\caption{\label{PrefilterLogLog}  Frequency  dependent windowing for the
normal.drc sample settings on the time-frequency plane. Logarithmic time
and frequency scales. The X axis is  time in milliseconds, the Y axis is
frequency in Hz.}
\end{center}
\end{figure}

\begin{figure}
\begin{center}
\includegraphics[width=\basepctwidth\textwidth,keepaspectratio]{figures/DBP-Gabor}
\caption{\label{Gabor} Frequency  dependent windowing for the normal.drc
sample  settings  on the  time-frequency   plane.  Logarithmic  time and
frequency scale  with Gabor  limit superimposed.  The X  axis is time in
milliseconds, the Y axis is frequency in Hz.}
\end{center}
\end{figure}

While developing DRC I've read some informal notes on the Internet
stating that on short time windows our perception of time should be
considered on a logarithmic scale too. I'm not quite convinced that this
assumption is actually true, but if such an assumption is correct our
perception of the room correction would be as in figures
\ref{PrefilterSemilogX} and \ref{PrefilterLogLog}. Even if this
assumption is not true these graphs are pretty useful to make clearly
visible the part of the time-frequency plane that gets corrected by DRC
and becomes even more useful if also the Gabor limit is placed in the
graph as in figure \ref{Gabor}.

The Gabor limit is defined by the following simple inequality:

\begin{displaymath}
	\Delta f \Delta t > \frac{1}{2}
\end{displaymath}
where $f$ is frequency and $t$ is time, and defines the limit of
uncertainity in the time-frequency plane. This means for example that,
looking at picture \ref{Gabor}, when the window exponent goes below
about $0.5$ the frequency dependent windowing starts violating the Gabor
inequality at least in some small frequency range. Within that range the
room transfer function estimation performed by DRC becomes inaccurate
and the room correction might be affected by appreciable errors in the
evaluation of the room transfer function.

\begin{figure}
\begin{center}
\includegraphics[width=\basepctwidth\textwidth,keepaspectratio]{figures/DBP-BPComparison}
\caption{\label{BPComparison}        Comparison   between  the  standard
proportional  windowing  curve  and the  new one  based on the  bilinear
transformation. Logarithmic time and frequency scale. The X axis is time
in milliseconds, the Y axis is frequency in Hz.}
\end{center}
\end{figure}

Starting  with  version  2.6.0 a  prefiltering  curve based  on the
bilinear transform has been introduced. This curve provides a better
match with the typical  resolution of the ear and  also with the typical
behaviour of the  listening room. The  windowing  curve provides the
same exact  results  as the  previous  windowing  curve when  the window
exponent is  set to 1.0,  but provides  a different  behaviour  when the
window exponent is changed, as showed  in figure \ref{BPComparison}. The
closer approximation of  the ear behaviour is  clearly visible in figure
\ref{BandwidthComparison},  where it is shown  that using an appropriate
configuration of the windowing  parameters it is possible to get a close
to perfect match with the  ERB psychoacoustic  scale (see curves labeled
ERB and erb.drc, which overlap almost perfectly).

\subsubsection{Pre-echo truncation}
\label{Pre-echoTruncation}

Starting from version 2.7.0 this step is implicitely disabled.
Considering that excess phase inversion is performed simply by time
reversal of the excess phase component, pre-echo is implicitely limited
by the frequency dependent windowing performed on the excess phase. The
description of this step and the code performing it has been retained
because it could be used for experimental reasons, especially
considering that the reequalization to flat performed in this step is
time reversed with respect to that performed in the excess phase
pre-processing. This meas that a minimum phase reequalization performed
in the pre-echo truncation is equivalent to a maximum phase
reequalization in the excess phase pre-processing, and the other way
around.

If enabled frequency dependent windowing is used to truncate the
pre-echo caused by the inversion of the excess phase part of the impulse
response. The truncation procedure is always the same but it is
implicitely applied on the left side of the time-frequency plane instead
of the right side, because inversion of the excess phase corresponds to
a time reversal. A much shorter windowing is used because our ear is
quite sensitive to pre-echo.

Windowing out part of the impulse response of the excess phase component
of the correction filter of course makes it no longer an all-pass
filter, i.e. the excess phase part no longer has a flat magnitude
response. To compensate for this problem the excess phase component
magnitude response is equalized back to flat using, after inversion, a
minimum phase filter. This of course causes some post-ringing. Even
though this usually happens only on few narrow bands, it might be quite
audible, and gets limited by the subsequent ringing truncation stage.

\subsubsection{Ringing truncation stage}
\label{RingingTruncationStage}

Since version 2.5.0 a further frequency dependent windowing is applied
directly to the filter after impulse response inversion and peak
limiting. This is performed to remove any residual ringing caused by the
previous steps, especially the dip and peak limiting steps, even though
this implies some tradeoff on filter accuracy.

\begin{figure}
\begin{center}
\includegraphics[width=\basepctwidth\textwidth,keepaspectratio]{figures/DBP-Jail}
\caption{\label{FDWJail}   Frequency  dependent  windowing  jail for the
normal.drc  sample  settings on the  time-frequency  plane.  Linear time
scale  and   logarithmic    frequency   scale.  The X  axis  is  time in
milliseconds, the Y axis is frequency in Hz.}
\end{center}
\end{figure}

With this further step the filter impulse response gets enclosed in a
sort of time-frequency jail defined by the excess phase windowing
settings on the left side of the time-frequency plane and by the ringing
truncation settings on the right side (see figure \ref{FDWJail}).
Considering that this time-frequency bounds have also some
psychoacoustic implications, with this time-frequency enclosure DRC
should be able to truncate automatically any part of the correction that
is probably going to cause audible artifacts. Following this lines DRC
gains at least a bit of psychoacoustically based ``self tuning'' and
should become more robust and less prone to artifacts.

\begin{figure}
\begin{center}
\includegraphics[width=\basepctwidth\textwidth,keepaspectratio]{figures/DBP-BWidthCmp}
\caption{\label{BandwidthComparison} Resolution bandwidth, as a function
of frequency, for the frequency dependent windowing and various standard
smoothing procedures, including the  Bark and ERB psychoacoustic scales.
The X axis  is frequency  in kHz,  the Y axis  is frequency  in Hz, both
plotted  on a   logarithmic  scale.  The  windowing   parameters  of the
normal.drc and erb.drc sample settings  files have been used to plot the
DRC resolution curves.}
\end{center}
\end{figure}

Applying  the Gabor  inequality  to the  window  length between  the two
curves of pre-echo  and ringing  truncation it is pretty  easy to get an
equivalent frequency  resolution, as a function  of center frequency, of
the frequency  dependent windowing  procedure. This  resolution could be
compared,  as in  figure  \ref{BandwidthComparison},   to some  standard
smoothing procedures  widely used  within many audio  applications, like
fractional     octave   smoothing   and  the   classical  Bark  and  ERB
psychoacoustic scales.

From figure    \ref{BandwidthComparison}  it  is pretty  clear  that the
correction  resolution  used  by DRC is  well above  that of  any of the
standard  smoothing  procedures,  at least  with the  normal.drc  sample
settings file (see section  \ref{SampleConfigurationFiles}).  This means
that the correction should  provide a perceived  frequency response that
is really close to the configured target frequency response.

The ``erb.drc'' resolution plot show  the approximation of the ERB scale
provided   by the   ``erb.drc''   sample   settings  file  (see  section
\ref{SampleConfigurationFiles}.    The  approximation  has  been created
assuming:

\begin{displaymath}
	\Delta f \Delta t = 2
\end{displaymath}
instead of the usual Gabor inequality (see figure \ref{Gabor}), i.e.
assuming that the frequency dependent windowing with the settings used
has a resolution that is about four times above the Gabor limit. This is
a rough estimate of the true resolution achieved by the DRC procedure in
this situation. This estimation has been derived considering the
compound effect of the various overlapped windows applied at various
stages of the filter generation procedure.

\subsection{Psychoacoustic target computation}
\label{PsychoacousticTargetComputation}

Starting from version 3.0.0 an optional stage, used to compute a target
frequency response based on the psychoacoustic perception of the
corrected frequency response, has been introduced. The target response
is based on the computation of the magnitude response envelope of the
corrected impulse response, which is an extension of the spectral
envelope concept. This is performed before the application of the usual
target response, so that the standard target response is not compensated
back by this stage.

The spectral envelope is a concept which has been introduced in the
field of speech synthesis and analysis and is defined simply as a smooth
curve connecting or somewhat following the peaks of the signal spectrum.
There are strong arguments and experimental evidence supporting this
approach and the idea that our ear uses the spectral envelope for the
recognition of sounds. The spectral envelope, for example, allow our ear
to understand speech under many different conditions, whether it is
voiced, whispered or generated by other means. These different
conditions generate completely different spectrums but usually pretty
similar spectral envelopes. The spectral envelope also easily explains
why our ear is more sensitive to peak in the magnitude response and less
sensitive to dips. A curve based on the peaks of the magnitude response
is by definition little or not affected at all by dips in the frequency
response.

In the speech recognition field many procedures have been developed to
compute the spectral envelope. Some of them are based on Linear
Predictive Coding (LPC), the Discrete Cepstrum, the so called ``True
Envelope'' and finally the Minimum Variance Distortionless Response
(MVDR). Most of these methods are optimized for speed, noise resilience
and to provide good results in the voice spectrum range sampled at low
sample rates, so they are not really suited for HiFi usage.

Within DRC a different procedure has been developed. This is a variation
of the usual fractional octave smoothing procedure, using the parametric
H\"older mean instead of the usual simple averaging. Furthermore the
smoothing has been extended to provide the Bark and ERB scales
resolution when applicable.

\begin{figure}
\begin{center}
\includegraphics[width=\basepctwidth\textwidth,keepaspectratio]{figures/DBP-SpectralEnvelope}
\caption{\label{SpectralEnvelope} Example of a magnitude response envelope. The
unsmoothed magnitude response of a typical room, corrected with a flat
target, is plotted. Superimposed there is the usual smoothing, computed
on the ERB scale, showing an essentially flat magnitude response, as
expected. The magnitude response envelope, computed on the ERB scale too using
standard parameters, show a rising slope which is in good agreement with
the inverse of one of the target responses suggested in the literature.}
\end{center}
\end{figure}

For the tuning of the parameters used in the magnitude response envelope
computation some typical real world room magnitude responses have been
taken. The computation parameters have been set so that the resulting
magnitude response envelope provides a target response as close as possible to the
inverse of the usual target responses suggested in the literature. An
example is reported in picture \ref{SpectralEnvelope}. The same
parameters have been tested also in some not so common room to check
that they were still providing the expected results. Of course,
considering that now the basic target response is provided by the
inverse of the magnitude response envelope, the usual target responses are no
longer needed, apart from subsonic or ultrasonic filtering, and should
be set to flat. For this reason the flat response is now the default
target for all standard configuration files. The standard target
response stage should be used only to adjust the response to taste, but,
unlike previous versions, for a neutral reproduction a flat response
should be used.

From the subjective point of view a system equalized to the inverse of
the magnitude response envelope usually sound really neutral. Even though most of
the times the magnitude response envelope response is different among the various
channels, resulting in an obvious channel misalignement if evaluated
with the standard smoothing procedures, imaging usually improves,
becoming more stable and focused. This appear to confirm that the
estimation performed by the magnitude response envelope has to be close to
the subjective perception of the magnitude response.

% Misura della risposta all'impulso
\subsection{Impulse response measurement}
\label{ImpulseResponseMeasurement}

Starting from  version  2.4.1 two  simple command  lines  tools, glsweep
(Generate Log Sweep) and  lsconv (Log Sweep  Convolution), are available
to perform  accurate time  aligned impulse  response  measurements. These
tools  are  based   on the  log  sweep   method  for  impulse   response
measurement, which is one of the most  accurate, especially for acoustic
measurements.  This  method is  based on  a special  signal,  which is a
logarithmic  sinusoidal sweep,  that need  to be reproduced  through the
system under test, and an inverse filter, which, when convolved with the
measured log sweep, gives back the impulse response of the system.

The steps needed to get the impulse response are the following:

\begin{enumerate}

\item  Generate   the log  sweep  and  inverse   filter  using  glsweep,
optionally converting the sweep to a suitable format.

\item Play the  log sweep  through your  system using a  soundcard while
recording the speaker output using a (hopefully good) microphone and any
recording program.

\item Convolve  the  recorded log  sweep with  the inverse  filter using
lsconv to get the final impulse response.

\end{enumerate}
If full duplex is supported by the  soundcard, recording may be performed
using  the  same  soundcard   used for   playing.  Using  two  different
soundcards or  a CD  Player  for  reproduction  and  a soundcard  for
recording  usually provides  worse results  unless they  are accurately
time synchronized.

The lsconv tool allows also for the use of a secondary reference channel
to correct  for  the  soundcard  frequency   response  and for  any time
misalignment caused  by the soundcard  itself, the  soundcard drivers or
the play and recording  programs. This soundcard  compensation of course
works if the reference  channel has the same  behaviour of the measuring
channel. The typical use of this feature would be the use of one channel
of a stereo or multichannel soundcard  to measure the system and another
one used in a loopback configuration to get the reference channel needed
to correct the soundcard itself.

With this configuration even with  cheap soundcards it is pretty easy to
get a $ \pm 0.1 $ dB  frequency response over the  audio frequency range
with near to perfect time alignment and phase response. Considering that
the log sweep method ensure by itself a strong noise rejection (90 dB of
S/N ratio is easily achievable even  in not so quiet environments) and a
strong   rejection  to   artifacts  caused  by  the  system  non  linear
distortions, with this  method the final measurements  usually have true
state of the art accuracy.

Finally  an  important  warning:  playing  the  log sweep  signal  at an
excessive   level can  easily   damage  your  speakers,  especially  the
tweeters. So be really  careful when playing such  a signal through your
equipment. The ``Jones  Rush'' guide (see section  \ref{JonesRushGuide})
provides some useful hints to help  you in the use of this kind of tools
without the risk of damaging your  equipment. No responsibility is taken
for any damage to your equipment, everything is at your own risk.

\subsubsection{The glsweep program}

When executed without parameters the glsweep program gives the following
output:

\begin{quote}
{\scriptsize
\begin{verbatim}
GLSweep 1.0.2: log sweep and inverse filter generation.
Copyright (C) 2002-2005 Denis Sbragion

Compiled with single precision arithmetic.

This program may be freely redistributed under the terms of
the GNU GPL and is provided to you as is, without any warranty
of any kind. Please read the file "COPYING" for details.

Usage: glsweep rate amplitude hzstart hzend duration silence
        leadin leadout sweepfile inversefile

Parameters:

  rate: reference sample rate
  amplitude: sweep amplitude
  hzstart: sweep start frequency
  hzend: sweep end frequency
  duration: sweep duration in seconds
  silence: leading and trailing silence duration in seconds
  leadin: leading window length as a fraction of duration
  leadout: trailing window length as a fraction of duration
  sweepfile: sweep file name
  inversefile: inverse sweep file name

Example: glsweep 44100 0.5 10 21000 45 2 0.05 0.005 sweep.pcm inverse.pcm
\end{verbatim}
}
\end{quote}
The program output contains some brief explanation of the generation
parameters and some sample options.

The longer the log  sweep used the  stronger the noise  rejection of the
measure. A 45 seconds log sweep usually  gives more than 90 dB of signal
to noise ratio in the final impulse  response even when used in somewhat
noisy environments,  for example one  where the computer  used to do the
measure is in the same room of the measured system, producing all of its
fan noise.  The output  format is  the usual raw  file with  32 bit IEEE
floating point samples. If you need to convert the sweep generated using
the example  above  to a 16 bit  mono WAV  file you  can use  SoX with a
command line like this one:

\begin{quote}
{\scriptsize
\begin{verbatim}
sox -t f32 -r 44100 -c 1 sweep.pcm -t wav -c 1 sweep.wav
\end{verbatim}
}
\end{quote}
SoX, can be downloaded at:

\begin{quote}
 \url{http://sox.sourceforge.net/}
\end{quote}
If you  want to  create a  stereo  WAV file  to get  also the  reference
channel you can use something like:

\begin{quote}
{\scriptsize
\begin{verbatim}
sox -t f32 -r 44100 -c 1 sweep.pcm -t wav -c 2 sweep.wav
\end{verbatim}
}
\end{quote}
The inverse filter  doesn't need to  be converted to a  WAV file because
lsconv is  already  able  to read  it as is.  If you  need to  convert a
recorded sweep stored in a wav file  back to a raw 32 bit floating point
format use this command:

\begin{quote}
{\scriptsize
\begin{verbatim}
sox recorded.wav -t f32 recorded.pcm
\end{verbatim}
}
\end{quote}
If you have a stereo WAV file with  both the measurement channel and the
reference channel you can extract them into two files using:

\begin{quote}
{\scriptsize
\begin{verbatim}
sox recorded.wav -t f32 -c 1 recorded.pcm mixer -l
sox recorded.wav -t f32 -c 1 reference.pcm mixer -r
\end{verbatim}
}
\end{quote}
provided   that the   recorded  channel  is  the  left one  (``mixer -l''
parameter)  and  the reference   channel is  the right  one  (``mixer -r''
parameter).

\subsubsection{The lsconv program}

When executed without parameters the  lsconv program gives the following
output:

\begin{quote}
{\scriptsize
\begin{verbatim}
LSConv 1.0.3: log sweep and inverse filter convolution.
Copyright (C) 2002-2005 Denis Sbragion

Compiled with single precision arithmetic.

This program may be freely redistributed under the terms of
the GNU GPL and is provided to you as is, without any warranty
of any kind. Please read the file "COPYING" for details.

Usage: LSConv sweepfile inversefile outfile [refsweep mingain [dlstart]]

Parameters:

  sweepfile: sweep file name
  inversefile: inverse sweep file name
  outfile: output impulse response file
  refsweep: reference channel sweep file name
  mingain: min gain for reference channel inversion
  dlstart: dip limiting start for reference channel inversion

Example: lsconv sweep.pcm inverse.pcm impulse.pcm refchannel.pcm 0.1 0.8
\end{verbatim}
}
\end{quote}
All files must be in the usual raw  32 bit floating point format. To get
the impulse  response without  the use of a  reference  channel just use
something like:

\begin{quote}
{\scriptsize
\begin{verbatim}
lsconv recorded.pcm inverse.pcm impulse.pcm
\end{verbatim}
}
\end{quote}
Where  ``recorded.pcm'' is  the recorded  sweep, ``inverse.pcm''  is the
inverse filter  generated by glsweep  and ``impulse.pcm''  is the output
impulse response ready to be fed to DRC.

If you also want to use the reference channel use something like:

\begin{quote}
{\scriptsize
\begin{verbatim}
lsconv recorded.pcm inverse.pcm impulse.pcm reference.pcm 0.1
\end{verbatim}
}
\end{quote}
The ``0.1'' value is the minimum  allowed gain for the reference channel
inversion. $0.1$ is the  same as -20 dB, i.e. no  more then 20 dB of the
reference channel frequency  response will be  corrected. This is needed
also to prevent  numerical  instabilities  caused by the  strong cut off
provided by the soundcard DAC and ADC brick wall filters.

When used  with the  reference  channel  the main  spike of  the impulse
response is  always at exactly  the same  length of the  log sweep used,
provided that the two soundcard  channels are perfectly synchronized. Of
course this is usually true for all soundcards.

For example if a 10 seconds sweep is used the main spike will be exactly
at 10 seconds from the beginning of the output impulse response, i.e. at
sample 441000 if a 44.1 kHz sample rate is used.

If the main spike is at a different  position it means that there's some
delay in the measurement  channel, usually caused  by the time the sound
takes to travel  from the  speaker to the  microphone. If  this delay is
different   for  different   channels   it means   that  there's a  time
misalignment   between  channels  that needs  to be  corrected.  Up to a
limited amount,  and  using some  small tricks,  DRC is already  able to
compensate          for      interchannel      delays    (see    section
\ref{InterchannelTimeAlignment}).  Some future  DRC release will include
better support for interchannel time alignment.

\subsubsection{Sample automated script file}

Under the ``source/contrib/Measure''  directory  of DRC there's a sample
Linux shell script, called  ``measure'', that  uses glsweep, lsconv, SoX
and standard ALSA play and recording  tools to automate the time aligned
measurement procedure using a reference  channel. This sample script can
be used only  under Linux and  it is just a  quick hack  to allow expert
users  to  automate  the  whole  procedure.  Use  it at  your own  risk.

Furthermore this script has been developed with an old version of SoX
(12.17.7) so it might need some changes to work with more recent
versions. It also needs any related tools ready to be executed from the
working directory, else it doesn't work. When executed without
parameters the script gives the following output:

\begin{quote}
{\scriptsize
\begin{verbatim}
Automatic measuring script.
Copyright (C) 2002-2005 Denis Sbragion

This program may be freely redistributed under the terms of
the GNU GPL and is provided to you as is, without any warranty
of any kind. Please read the file COPYING for details.

Usage:
 measure bits rate startf endf lslen lssil indev outdev impfile [sweepfile]

 bits: measuring bits (16 or 24)
 rate: sample rate
 startf: sweep start frequency in Hz
 endf: sweep end frequency in Hz
 lslen: log sweep length in seconds
 lssil: log sweep silence length in seconds
 indev: ALSA input device
 outdev: ALSA output device
 impfile: impulse response output file
 sweepfile: optional wav file name for the recorded sweep

example: measure 16 44100 5 21000 45 2 plughw plughw impulse.pcm
\end{verbatim}
}
\end{quote}
This script assumes  that the measuring  channel is on  the left channel
and that the reference  channel is the right one.  To use it just take a
look at the  sample  command line  provided  above. You have  to provide
proper ALSA input and output devices,  but ``plughw'' usually works with
most soundcards.

Using 24 bits of  resolution to measure  an impulse  response is usually
just a waste of resources.  In most rooms getting  a recorded sweep with
more than  60 dB of  S/N ratio  is close  to impossible,  so  16 bits of
resolution are already plain overkill.  On the other hand, thanks to the
strong noise  rejection  provided by the log  sweep method,  a sweep S/N
ratio of 60  dB is already  high  enough to get  more than  90 dB of S/N
ratio in the  recovered  impulse  response, at  least with a  45 s sweep
running at a 44.1 kHz sample rate.

The impulse response is what DRC works on, so it is the impulse response
that needs an high S/N ratio, not the sweep. If you really want a better
impulse response  S/N ratio, or if  you measure in a  noisy environment,
increase the  sweep  length instead  of using  24 bits of  resolution. A
longer  sweep  will  improve  the S/N  ratio of  the  impulse  response,
increasing the resolution instead will provide no benefit at all.

Chris Birkinshaw created a modified  version of the measure script which
adds Jack support. The script is named  ``measurejack'' and you can find
it under the  ``source/contrib/MeasureJack''  directory  of the standard
distribution. For informations about Jack take a look at:

\begin{quote}
\url{http://jackit.sourceforge.net/}
\end{quote}
Finally  Ed Wildgoose   created a  simple  program  with about  the same
functionality  of the measure  script. It  works also under  Windows and
being written  in C, instead  of being a  simple shell  script, it is less
dependent   on  other  tools  and  usually   provides  a  more  reliable
functionality. You can download it from:

\begin{quote}
\sloppy\raggedright
\url{http://www.duffroomcorrection.com/wiki/Simple_Automated_IR_Measuring_Tool}
\end{quote}

\subsubsection{Beware cheap, resampling, soundcards}

Most  cheap  game  oriented   soundcards  often  include  a sample  rate
converter in their  design, so that  input streams  running at different
sample rates can  be played together  by resampling  them at the maximum
sample rate supported  by the soundcard  DAC. Usually  this is 48 kHz as
defined by the AC97 standard. These  sample rate converters often are of
abysmal quality, causing all sort of aliasing artifacts.

Most deconvolution based impulse response measurement methods, including
the log sweep method, are quite robust  and noise insensitive, but cause
all sorts of  artifacts  when non  harmonic but signal  related
distortion  is  introduced,   even at  quite low  levels.  The  aliasing
artifacts introduced by  low quality sample rate  converters are exactly
of this  kind and  are one  of the  most  common cause  of poor  quality
impulse response measurements and consequently of correction artifacts.

\subsubsection{How to work around your cheap, resampling, soundcard}

Despite this, most of the times good  measurements are possible even out
of cheap soundcards if  the maximum sample rate  supported by the DAC is
used,  usually  48 kHz,  so  that the  soundcard  internal  sample  rate
converter isn't used at all. You can  change the impulse response sample
rate after  the  measurement  using high  quality  software  sample rate
conversion  algorithms  (see  section   \ref{SampleRateConversion}), thus
preserving the impulse response quality.

To check  the quality  of the  impulse  response  measurement  perform a
loopback   measurement,  without  using  a reference   channel  else any
measurement   problem  will  be  washed  out by  the  reference  channel
compensation. The impulse response  you get must be a single clean spike
much similar to that of a CD Player  (see for example the upper graph of
picture     \ref{BaselineImpulseResponseFullRange},     labeled  ``Dirac
delta''). A bit of ringing before and/or after the main spike is normal,
but anything else is just an artifact.  Only when you are sure that the
measurement chain is  working as expected open  the loopback and do the
real  measurement,   eventually   adding  also a  reference   channel to
compensate for any remaining soundcard anomaly.

\subsection{Sample rate conversion}
\label{SampleRateConversion}

If you have the impulse response sampled at a different rate than the
one needed for the final filter, you need to convert the sample rate
before creating or applying the filters. For example you might have a 48
kHz impulse response but you may need to filter standard CD output at
44.1 kHz. In this situation you can either convert the impulse response
to 44.1 kHz before feeding it to DRC or you can convert the resulting
filters to 44.1 kHz after DRC has created them. I generally prefer the
first procedure, which leads to exact filter lengths in the DRC final
windowing stage, but in both cases you need a good quality sample rate
converter, which uses, for example, band limited interpolation. A reasonable
choice, free both under Linux and Win32, is SoX, which may be downloaded
at:

\begin{quote}
\url{http://sox.sourceforge.net/}
\end{quote}
Recent versions of SoX include some top quality sample rate conversion routines.
SoX also provides a lot of other features for sound files manipulation.
For a reference on band limited interpolation take a look at:

\begin{quote}
\url{http://ccrma-www.stanford.edu/~jos/resample/}
\end{quote}
Another free good sample rate converter comes from the shibatch audio
tools suite. This sample rate converter provides a quality which is
adequate for the task of converting the impulse response file before
feeding it to DRC. You can find the shibatch audio tools at it at:

\begin{quote}
\url{http://shibatch.sourceforge.net/}
\end{quote}

\subsection{Correction tuning}
\label{CorrectionTuning}

Proper  tuning of the  correction  filter  generation  procedure  easily
provides   a  substantial    improvement   over  the  standard   sample
configuration      files    provided   along   with  DRC  (see   section
\ref{SampleConfigurationFiles}). To properly tune the filters to closely
match your room behaviour there are many different issues that should be
taken into account.

\subsubsection{Preventing pre-echo artifacts}
\label{AvoidPreEchoArtifacts}

One of  the main  problems  in  digital  room  correction  are  pre-echo
artifacts  that  arise when   compensation  accuracy  is pushed  above a
certain threshold. This pre-echo artifacts usually occur on narrow bands
and are easily audible as a sort of  ringing or garble before transients
or sharp attacks. In order to avoid them there are basicly two options:

\begin{itemize}

\item Reduce the correction on critical frequency regions where pre-echo
artifacts may arise.

\item Use a minimum phase approach to avoid pre-echoes. This way you get
increased ringing after the main spike  instead of pre-echo, but this is
usually masked both by  our ear temporal masking  and by the reverberant
nature of common listening rooms, so it is much less audible, if audible
at all.

\end{itemize}
DRC uses both options in different steps of the correction procedure. So
in order to avoid pre-echo artifacts you basically have to:

\begin{itemize}
\item Reduce the amount of correction applied to the excess phase
component by reducing the size of the frequency dependent window
applied. With the standard configuration files this is usually
everything that need to be done, because all the other procedures are
already configured to avoid pre-echo problems.

\item Use  a long  enough FFT  where  circular  convolution  is involved
(basicly homomorphic  deconvolution and pre-echo  truncation inversion),
because circular artifacts may easily become pre-echo.

\item Use the single side  sliding lowpass  prefiltering procedure; this
is just because of small numerical  errors in band windowing that causes
small amounts of pre-echo on band edges.

\item Use  the  minimum  phase  versions for  some of  the  accompanying
procedures (peak and dip limiting for example)

\item Use the pre-echo truncation fast deconvolution for the inversion
procedure, with appropriate pre-echo truncation parameters. By the way,
if the excess phase component windowing parameters are already set
appropriately, this should not be needed.

\end{itemize}
The sample configuration files supplied  are a good example of all these
options combined together. In normal situations you can use them as they
are changing  only EPLowerWindow,   EPWindowExponent,  MPLowerWindow and
MPWindowExponent  to fit your needs.

\subsubsection{Preventing clipping}
\label{PreventingClipping}

One of the problems of real time correction is the prevention of DAC
clipping caused by the filter intrinsic amplification. First of all the
normalization factor to be used (see sections \ref{PSNormFactor} and
\ref{MCNormFactor}) depends on the convolver used. Some convolvers want
the filter normalized to the 16 bit range, i.e. 32768, most others want
a standard normalization, i.e. normalization to $\pm 1.0$. For example
BruteFIR needs a filter normalized to 1.0 to get 0 dB amplification
between input and output.

All the normalization  steps used  within DRC, included  those needed to
output   the  final   filter   (see   sections    \ref{PSNormType}   and
\ref{MCNormType}), accept four types of normalization:

\begin{itemize}
	\item S, i.e. sum normalization, also called $L_1$ norm
	\item E, i.e. euclidean normalization, also called $L_2$ norm
	\item M, i.e. Max normalization, also called $L_\infty$ norm
	\item P, i.e. Peak normalization, i.e. normalization to the highest amplitude response peak
\end{itemize}
For a  detailed  description  of the four types  of normalization  see
section \ref{BCNormType}.

The S normalization guarantees against overflows in the output stream,
i.e. it guarantees that if any input sample is never greater than X than
any output sample is never greater than X multiplied by the
normalization factor. This means also that if the normalization factor
is 1 and the input sample is never greater than 32767 (i.e. the input is
a 16 bit stream) the output is never greater than 32767, i.e. a 16 bit
DAC on output will never clip or overflow.

Anyway, using common musical signals, and depending on the filter
frequency response, the use of the S normalization might lead to filters
with a global gain substantially lower than 1 (0 dB), i.e. filters with
a typical output level which is lower, most of the times much lower,
than the input level. With such low levels part of the resolution of the
DAC gets lost. With normal musical signals it is usually safe to use a
filter with an S normalization factor greater than 1, because,
considering the typical frequency response of a room, and the
corresponding reversed frequency response of the filter, overflows would
occur in frequency ranges where typically there is not enough musical
signal to cause it.

A good extimation of an adequate normalization factor might be provided
by the P normalization, which rescales the filter so that the highest
peak in the magnitude response corresponds to a 0 dB gain. Contrary to
intuition this doesn't completely ensure that there would never be
clipping on output, but for most common musical signals it usually
provide a safe extimation without introducing excessive attenuation.

If you use BruteFIR one possible approach is to use 1 for the
PSNormFactor and S for PSNormType and then use the rescaling and
monitoring features of BruteFIR to boost the gain up to few dB below
overflow with typical musical signal. You might try using a 0 dB white
noise source as a sort of worst case situation.

Be careful setting the basic filter gain. I found that many recent
musical recordings, especially compressed and rescaled pop music
productions, cause output levels that are just 1 or 2 dB below the white
noise worst case scenario. The degradation in the sound quality caused
by DAC clipping is typically much more audible than the degradation you
get loosing a single bit or less of your DAC resolution, especially if
you use 16 bit DACs along with dithering or 24 bits DACs.

If you're unable to perform a test using 0 dB white noise as the input
signal and for some reason you don't want to rely on the estimation
performed by the P normalization, a simple rule of thumb is to use the E
normalization with a normalization factor which is a couple of dB lower
than the maximum gain allowed during peak limiting. With the standard
configuration files, where the maximum allowed gain is never greater
than about 6 dB, this means using a normalization factor around $0.3 -
0.4$ with convolvers which use 1.0 as the 0 dB reference level like
BruteFIR, or using something like 10000 - 13000 with convolvers which
use 32768 as the 0 dB reference level.

The default configuration files all use the E normalization with a
normalization factor set to 1, leaving the task of scaling the filter gain
to avoid clipping to the convolver. With the default configuration
files you should set the convolver gain to something below $-6 dB$, which
is the default maximum gain allowed by the peak limiting procedure.

\subsubsection{Some notes about loudspeaker placement}

As most audio practitioneers already know, in a basic stereo loudspeaker
configuration it is important that the distance between the loudspeakers
and the listening position is exactly the same for both loudspeakers,
and also not too different from the distance between the two
loudspeakers themselves (the classical equilateral triangle placement).
If this rule isn't satisfied usually the stereo image become distorted
and confused. With digital room correction enabled this rule becomes of
paramount importance.

DRC doesn't automatically compensate for delays caused by loudspeaker
misplacement and having the two channels with near to perfect direct
sound, both in phase and magnitude, makes any difference in the arrival
time immediately and clearly audible, with a nasty phasey sound and a
blurred stereo image. Less than 10 cm are enough to cause clearly
audible problems, so take your time to measure the distance from both
loudspeakers and the listening position before doing any measure, and
also do your measures by placing the microphone exactly at the listening
position.

Furthermore, with digital room correction it is worth to experiment with
unusual speaker placements. Reflections from nearby walls are more
difficult to correct when they are away from the main spike, so placing
the speakers near to the walls, or may be even in the corners, might
sometime give better results with DRC, provided that you place some
absorbing material near the speakers to remove early reflections in the
high frequency range, where DRC is able to correct only the direct
sound.

This type of placement  is exactly the opposite  of what is usually done
if you don't use digital room correction,  where it is usually better to
try to put loudspeakers away from the  walls to avoid early reflections,
that cause major  problems to the  sound reproduction  and almost always
boomy bass.  Anyway, remember  that there  is no ready to  use recipe to
find the  best speaker  placement,  even  with DRC  in use,  so a bit of
experimenting is always needed.

\subsubsection{Some notes about channel balance}

DRC doesn't compensate  for channel  level imbalance,  so this should be
done manually after correction by changing a little the filters level until
a perfect balance is  achieved. This is of course  better achieved using
an SPL  meter  with  pink  noise  and  proper  weighting.  Anyway  after
correction the two  channels start  having a frequency  response that is
pretty much the same, so  achieving perfect balance  becomes pretty easy
even by ear. Just use a mono male,  or, better, female voice, and adjust
the filters level until the voice  comes exactly from the center of both
loudspeakers.

To achieve a good balance you might also use the level hints provided by
DRC at the end of the correction procedure, provided that the measured
impulse responses have levels that are directly related to the original
levels of the channels, i.e. these levels haven't been changed by the
measuring procedure itself.

\subsubsection{Interchannel time alignment}
\label{InterchannelTimeAlignment}

First  of all  the  current  DRC   release  is able  to  compensate  for
interchannel  misalignment of only  few samples, no  more than $ \pm 8 $
with the default configuration files.  Furthermore accurate time aligned
measurements must be supplied, using either the glsweep and lsconv tools
with a reference channel or some other tool providing the same degree of
accuracy.

To get this  limited time  alignment you  have to execute  the following
steps:

\begin{enumerate}

\item Execute DRC on one  channel as usual. At  the beginning of the DRC
output on screen you will see a line like this one:

\begin{quote}
\begin{verbatim}
Impulse center found at sample 1367280.
\end{verbatim}
\end{quote}
Take note of the impulse center value.

\item After DRC  has finished  prepare the  configuration  files for the
other channels as usual but change  the BCImpulseCenterMode parameter to
M and the BCImpulseCenter  parameter from 0 to  the value of the impulse
center noted  before for the  first channel.  This way DRC  will use the
value of the impulse center of the  first channel as a reference for the
other channels and will compensate  for any misalignment with respect to
the first channel. If channels are misaligned more than few samples this
will cause errors  in the correction  filters, usually  causing a rising
frequency response, and so a bright sound.

\end{enumerate}

\subsubsection{How to tune the filters for your audio system}
\label{HowToTuneFilters}

A proper tuning of the filters for  your audio system and your listening
room  easily  provides  a  substantial  improvement   over the  standard
configuration files.  The best way  to do this is to  use the correction
simulation  provided by DRC  and to check  the results  using the Octave
scripts       supplied    with   the    documentation    (see    section
\ref{SampleResults}),      but  if  you  have  little   experience  with
measurements interpretation  you can also try to  tune the correction by
simple listening to the results, even though it isn't an easy task.

One of the most common mistakes performed in the tuning procedure is the
use of an excessive correction, which  initially gives the impression of
a good  result,  but cause  also the  appearance  of  subtle  correction
artifacts that becomes audible only with some specific musical tracks.
These artifacts often have a peculiar  resonant behaviour so they become
audible only when they get excited  by specific signals. To learn how to
recognize them  try using the  ''insane.drc''  sample  configuration file,
which applies an overly excessive  amount of correction, causing clearly
audible artifacts on all but the most damped rooms.

The best  procedure  to  use  is to  start from  the  minimal  amount of
correction, like that provided by the  minimal.drc or erb.drc correction
settings. If  your impulse  response  measurements  are of  good quality
these   minimalistic   correction  settings  should  already   provide a
substantial   improvement  over  the  uncorrected  system,  without  any
perceivable artifact. If this doesn't happen it's better to first double
check the  measurements  performed before  fiddling with  the correction
parameters.  Remember  that measurements  problems  are the  most common
cause of unsatisfactory correction results.

After this  first  test you can  slowly  switch to  stronger  correction
settings using  the  soft.drc,  normal.drc,  strong.drc and  extreme.drc
settings, always listening  to the results after  each step, if possible
using quick  switching between  the filters.  When correction  artifacts
start to arise,  which usually happens  between the  normal.drc settings
and the extreme.drc  settings, it's  time to stop and  to start playing
with some specific correction parameters.

The first  parameters  to modify  are those  that define the  windowing
correction   curve   applied  to  the  signal,  i.e    MPWindowExponent,
EPWindowExponent, RTWindowExponent,  slowly
reducing  them to  0.95,  0.9, 0.85  and so  on, down  to about  0.7, thus
reducing the  correction in the  critical mid and  mid-bass range. These
are really sensitive  parameters, so changing  them by as little as 0.01
easily cause an audible difference, especially when you are close to the
boundary where correction artifacts  start to appear. When the artifacts
disappear  you can  start  increasing the  windows  applied  to the bass
range, slowly increasing,  by about a 5\% at a  time, the MPLowerWindow,
EPLowerWindow, RTLowerWindow  parameters,  until
artifacts start to appear  again. After that you  can decrease again the
window exponent parameters  until artifacts  disappear again, and so on.

This procedure may be repeated until there's no further improvement or
the parameters reach an excessive value, i.e below about 0.6 for the
window exponents, above 1 second for the minimum phase and ringing
truncation windowing parameters (MPLowerWindow, RTLowerWindow) and above
100 ms for the excess phase windowing parameter (EPLowerWindow).
Remember also to set the pre-echo truncation parameters (ISPELowerWindow,
ISPEUpperWindow) according to the excess phase windowing parameters (see
sections \ref{ISPELowerWindow} and \ref{ISPEUpperWindow}).

Of course  the tuning  procedure  has to  be carefully  adapted  to your
specific room, so,  after a good  tuning has been reached  following the
basic procedure, you can further try  playing a little with the available
parameters, applying even  different values to  each of them, proceeding
one at a time to avoid confusion. By  the way, be careful, because after
the initial tuning the differences  between the filters will start to be
quite  subtle,  most of the  times  will be  barely  audible,  and quick
switching between  the filters,  possibly even under  blind conditions,
will become almost mandatory  to really understand  what's happening and
which filter is better or at least audibly different.

\section{Program compilation and execution}

DRC can be  compiled  either under  Win32 or  Linux, but  because of its
simplicity  it will  probably work  under most  operating  systems with a
decent C++  compiler  with  support for  the standard  template  library
(STL). The Win32 executable  (drc.exe) is provided  precompiled with the
standard DRC  distribution under the  sample directory,  where there are
also  the   executables  for  the  impulse   response   measuring  tools
(glsweep.exe and lsconv.exe).

A Makefile is provided for Linux and other Unixes, but it has been
tested only under Fedora Core 11. To build the program under Linux
usually what you have to do is just type ``make'' in the source
directory of DRC, where the makefile resides. A Code::Blocks workspace
is also available for use both under Win32 and Linux. Code::Blocks can
be downloaded at:

\begin{quote}
\url{http://www.codeblocks.org/}
\end{quote}
The file drc.h contains  a configurable define  (UseDouble) which can be
set to use  double  or float  as the  data  type used  for all  internal
computations. Despite some microscopic  differences in the final output,
I have never  found any  real advantage  using  doubles as  the internal
basic type.

During the testing for the 3.1.0  release I have performed some tests to
check the  signal to  noise  ratio of the  output  filters.  Despite the
amount of processing performed and  the fact that little effort has been
placed into keeping the maximum accuracy throughout the processing, even
using single precision arithmetic the signal to noise ratio of the final
filter resulted  to be greater  than 140 dB  in the worst  case. This is
more than 20 dB  better than the  signal to noise ratio  provided by the
best DACs available in the world.

Considering these results  the supplied Win32  executable is compiled for
single precision. If you want to switch to the double precision you have
to recompile it yourself. Of course using the double data type makes DRC
a bit slower and, most important, much more memory intensive.

Starting from version 2.4.0 DRC is  able to use the Takuya Ooura and GNU
Scientific Library FFT  routines, which are  included in the distributed
package. The inclusion of these  routines is controlled by the UseGSLFft
and UseOouraFft  defines  in drc.h.  These  routines are about  10 times
faster than the standard routines  used by DRC, but to use them with the
STL complex   data type  a  clumsy hack  has  been used,  and it  is not
guaranteed that  this hack  will work with  all the STL  implementations
available. If it causes any problem simply comment out the UseGSLFft and
UseOouraFft defines in drc.h and  recompile the program. This will force
DRC to use the older, slower but STL compliant FFT routines.

The accuracy of the different FFT  routines is pretty much the same. The
Ooura routines work only with powers  of two lengths, so they are used only
with power of two  lengths computations.  Ooura FFT  routines are somewhat
faster than the GSL  routines but are  also a little  bit less accurate.
The default configuration uses only  the GSL FFT routines, providing the
best  compromise  between speed  and  accuracy. The  Ooura FFT  routines
become useful when DRC is compiled for double precision arithmetic.

Most text files  supplied with the  standard distribution  use Unix line
termination (LF instead  of CR/LF). Be aware of  this when opening files
under Win32 systems.  WordPad is able to open LF  terminated text files,
NotePad isn't.

Finally an important note, especially  for Win32 users. DRC is a console
program, it has no graphical interface. All program execution parameters
must reside in a plain ASCII text  file which is supplied as an argument
on the program command line. In order to execute the program you have to
open a command prompt (or  DOS Prompt or whatever  is named a console in
your version of Windows) and type something like:

\begin{quote}
\begin{verbatim}
drc test.drc
\end{verbatim}
\end{quote}
followed by a carriage return (enter  or return key). Test.drc should be
the text file  already prepared  with all  the parameters  needed to run
DRC.

Under Linux  of course  you  have to use  a console  program  (the Linux
console, a terminal emulator like XTerm or something like this if you are
using XWindows). The DRC executable must be in the system path or in the
directory where you execute.

\subsection{Command line parameters replacing}

Starting  from  version   2.6.2 all  the  parameters   available  in the
configuration  file may be  replaced by an  equivalent  parameter on the
command  line. For  example if  you want  just to  change the  input and
output filter  files of  the normal.drc  sample  configuration  file you
may use a command like this:

\begin{quote}
\begin{verbatim}
drc --BCInFile=myfile.pcm --PSOutFile=myfilter.pcm normal.drc
\end{verbatim}
\end{quote}
The parameter parsing  procedure supports also quoting of filenames with
spaces and  setting of  strings to  empty values,  which is  the same as
commenting a  parameter in  the configuration  file. For  example to use
some filename with spaces, to disable the output of the test convolution
file, to change the maximum allowed  gain, all in a custom configuration
file with spaces in its name, you could use a command line like this:

\begin{quote}
\begin{verbatim}
drc --BCInFile="my file.pcm" --PSOutFile="my filter.pcm"
  --TCOutFile="" --PLMaxGain=3.5 "my custom config.drc"
\end{verbatim}
\end{quote}
Along with  all the  default  configuration  parameters there  is also a
special   ``--help''  parameter   that  show the  full  list of  all the
available parameters with the associated parameter type. The list of the
parameters is really long, so some pager is needed to see them all.

\subsection{Sample configuration files}
\label{SampleConfigurationFiles}

DRC has started as an experimental program and because of this it has a
lot of tunable parameters, actually more than 150. Only few of them are
really important for the final filter correction quality. Most of them
are used just to take a look at intermediate results and check that
everything is working as expected. DRC flexibility might of course be
used also to deal with complex or unusual situations or to experiment
with weird configurations.

Along with the DRC distribution six main sample configuration files are
provided: minimal-XX.X.drc, soft-XX.X.drc, normal-XX.X.drc,
strong-XX.X.drc, extreme-XX.X.drc, insane-XX.X.drc. The ''XX.X' in the
file name stands for the sample rate in kHz which the file is configured
for. For example ''normal-44.1.drc'' is the normal configuration file
for the 44.1 kHz sample rate. Considering that the only difference
between the files is the base sample rate they are configured for, in
the rest of this document all files are named omitting the sample rate
part.

The sample configuration files provide most parameters set to reasonable
defaults, with stronger correction, but also worse listening position
sensitivity, going from the minimal.drc settings to the extreme.drc
settings. In the same directory of the configuration file there is also
a sample impulse response (rs.pcm, this is the impulse response of the
right channel of my previous HiFi system, in 32 bit IEEE raw format) usable
with the sample configuration files to see just what happens when DRC is
run.

The insane correction settings are not  meant for normal use but are used
just to provide  an example  of excessive  correction that  is going for
sure to cause audible correction artifacts. Using these settings file you
can easily check how  correction artifacts actually  sound like, thus
learning to identify them within normal filters  while you are tuning
them for your audio system (see section \ref{HowToTuneFilters}).

Another interesting sample configuration is the one provided by
the ``erb.drc'' file. This file provides an accurate approximation of
the ERB psychoacoustic scale (see figure \ref{BandwidthComparison}). It
is important to notice that basicly the correction isn't much stronger
than the ``minimal.drc'' sample configuration, but being approximately
tuned to our ear psychoacoustic resolution it is probably going to
provide a good perceived correction accuracy with minimal listening
position sensitivity, and so it is well suited for multiple listeners
situations, like home theater applications.

Starting form version 2.7.0 all sample configuration files are available
for 44.1 kHz, 48 kHz, 88.2 kHz, 96 kHz sample rates. By the way you
should be careful with sample rates above 44.1 kHz because most of this
sample files have been simply derived from the 44.1 kHz version without
testing them in real life situations. The sample configuration files for
the higher sample rates aren't in the sample directory. To avoid placing
a lot of similar files in the same directory the files for the higher
sample rates are placed in the ''src/config'' directory.

Remember also that all the sample correction files output the correction
filter (rps.pcm) in 32 bit floating point format normalized to 1.0,
which is the format suited for use with BruteFIR. Most sound editors
expect 16 bits integer files normalized to 32768, so the file above
might look either empty or completely clipped when opened with a sound
editor without using the appropriate options.

\subsubsection{Target magnitude response}
\label{TargetMagnitudeResponse}

Starting from version 3.0.0 the basic target magnitude response is
automatically generated by DRC using a specific procedure based on some
documented psychoacoustic assumptions. See section
\ref{PsychoacousticTargetComputation} for the details. Because of this
there should be no need to define a specific target curve and a flat
target, with just some limiting for subsonic and ultrasonic frequencies,
should be used instead. This is accomplished by using the pa-XX.X.txt
target, which is now the default for all standard configuration files
and is just a variation of the previous flat target adjusted to better
work with the ``B Spline'' target transfer function interpolation
procedure. The old target responses are retained for those situations
where the old approach may be preferable, and may be found under the
''source/target'' directory. Of course the postfiltering stage might be
used to adjust the magnitude response to taste, but for a neutral
reproduction the target magnitude response should be left to the flat
one.

\begin{figure}
\begin{center}
\includegraphics[width=\basepctwidth\textwidth,keepaspectratio]{figures/DBP-DTFCmp}
\caption{\label{DefaultTargetFunctions}   Comparison  of the main target
functions provided along with DRC. }
\end{center}
\end{figure}

The most important postfiltering target magnitude response files
supplied in the standard distribution (see section \ref{PSPointsFile}
``PSPointsFile'' for details) are subultra-XX.X.txt, bk-XX.X.txt,
bk-2-XX.X.txt, bk-3-XX.X.txt (see figure \ref{DefaultTargetFunctions}).
Here again ''XX.X'' stands for the sample rate used and is omitted in
the rest of this document. The target response files for the higher
sample rates are available in the ''src/target'' directory.

The first target response file provides just simple removal of
overcompensation on the extremes of the frequency range and has a linear
target frequency response, so it hasn't been plotted in figure
\ref{DefaultTargetFunctions}. The bk.txt file follows the Bruel \& Kjaer
(i.e.\ M{\oe}ller) recommendations for listening room frequency
response, i.e.\ linear from 20 Hz to 400 Hz, followed by a slow decrease
of 1 dB per octave up to 20 kHz. The bk-2.txt file is similar to bk.txt
but it is linear up to 200 Hz and then provides a slow tilt of $0.5$ dB
per octave up to 20 kHz. The bk-3.txt file is somewhat between bk-2.txt
and bk.txt, with a $0.5$ dB per octave tilt above $100$ Hz. The versions
with the ``sub'' suffix are the same target functions with the addition
of a steep subsonic filter. The versions with the ``spline'' suffix are
again the same target transfer functions but with a set of control
points suitable for the ``B Spline'' target transfer function
interpolation. Figure \ref{DefaultTargetFunctions} also show an example
of PCHIP interpolation of the "bk-3-sub" target function. In the sample
directory there are also some other simple postfiltering target files.

Like the sample configuration files starting from version 2.7.0 the
sample target responses are available for 44.1 kHz, 48 kHz, 88.2 kHz and
96 kHz sample rates and they must be used with the corresponding set of
configuration files. By the way the target response files for higher
sample rates are simply extended versions of the 44.1 kHz target
responses created by simply moving the last frequency point up to the
Nyquist frequency. This means that for most configuration files there is
either a gentle roll-off at higher frequency or a supersonic brickwall
filter applied above 20 kHz. If you want to properly correct content
above 20 kHz, provided that you have a microphone capable of recording
ultrasonic frequencies, you have to adapt the supplied files to your
needs.

\section{DRC Configuration file reference}

The DRC configuration file is a simple ASCII file with parameters in the
form:

\begin{verbatim}
ParamName = value
\end{verbatim}
Everything after a '{\tt \#}' and blank lines are considered comments and
are ignored. Each parameter has a two character prefix which defines the
step the parameter refers to. These prefixes are:

\begin{itemize}
\item BC = Base Configuration
\item MC = Microphone Compensation stage
\item HD = Homomorphic Deconvolution
\item MP = Minimum phase Prefiltering stage
\item DL = Dip Limiting stage
\item EP = Excess phase Prefiltering stage
\item PC = Prefiltering Completion stage
\item IS = Inversion Stage
\item PT = Psychoacoustic Target
\item PL = Peak Limiting
\item RT = Ringing Truncation stage
\item PS = Postfiltering Stage
\item MS = Minimum phase filter generation Stage
\item TC = Test Convolution stage
\end{itemize}
DRC does some checks to ensure that  each parameter provided has a value
that makes sense, but it isn't  bulletproof at all with respect to this.
Providing invalid or incorrect  parameters values may cause it to fail, or even
to crash.

Parameters which are important for  the quality of the generated filters
are marked with (*). When it makes  sense a reasonable value or range of
values is also provided, but the range supplied is always referred to the
44.1 kHz sample rate.

Many parameters have often a value which is a power of two. This is
mainly for performance reasons. Many steps require one or more FFT
computations, which are usually much faster with arrays whose length is
a power of two. The default values supplied are defined for a 44.1 kHz
sample rate. If a different sample rate is used the supplied values
should be scaled accordingly.

\subsection{BC - Base Configuration}

\subsubsection{BCBaseDir}
\label{BCBaseDir}

This parameter define the  base directory that  is prepended to all file
parameters, like for example BCInFile, HDMPOutFile or PSPointsFile. This
parameter allow the implicit definition of a library directory where all
DRC support  file  might  be placed.

File parameters  supplied on the  command line are not  affected by this
parameter unless the BCBaseDir parameter is also supplied on the command
line. File  parameters supplied  in the  configuration  file are instead
always affected  by the  BCBaseDir parameter,  no matter  if it has been
supplied in the configuration file or on the command line.

\subsubsection{BCInFile}
\label{BCInFile}

Just the name of the input file with the input room impulse response.

\subsubsection{BCInFileType}
\label{BCInFileType}

The type of the input file. D = Double, F = Float, I = Integer.

\subsubsection{BCSampleRate}
\label{BCSampleRate}

The sample rate of the input file. Usually 44100 or 48000.

\subsubsection{BCImpulseCenterMode}
\label{BCImpulseCenterMode}

The impulse   response  impulse  center may  be set  manually  using the
BCImpulseCenter    parameter   or you  may  ask DRC  to try  to  find it
automatically. If BCImpulseCenterMode  is set to A DRC will look for the
impulse center within the input file. If BCImpulseCenterMode is set to M
DRC uses the impulse center supplied with the BCImpulseCenter parameter.

Be careful  when using  automatic  impulse  center  recognition.  Strong
reflections or  weird situations  may easily  fool the  simple procedure
used by DRC, which simply looks for the sample with the maximum absolute
amplitude.

\subsubsection{BCImpulseCenter (*)}
\label{BCImpulseCenter}

This is the  position, in  samples, of the  time axis zero  of the impulse
response  read from  BCInFile.  Usually  this is  where the  peak of the
impulse response is, but for complex situations it might not be easy to identify
where the zero is. Even  few samples displacement  in this parameter may
cause high  frequency   overcorrection,   causing too  bright  sound. If
BCImpulseCenterMode is set to A this parameter is ignored.

\subsubsection{BCInitWindow}
\label{BCInitWindow}

Initial portion of the impulse response which is used to perform the
correction. It should be long enough to accomodate for any subsequent
windowing performed by DRC. The window is symmetrical with respect of
the impulse center. If needed, the signal is padded with zeroes. Usual
values are between 16384 and 131072, depending on the values of the
parameters for the subsequent steps. This initial window may be further
limited in subsequent steps, which sets the real window used.

\subsubsection{BCPreWindowLen}
\label{BCPreWindowLen}

This is the length of the window used to remove any noise coming before the
impulse center. This is  usually just few samples,  with a typical value
of 1024 samples, corresponding to 23.2  ms at a 44.1 kHz sample rate. If this
value is 0 this step is skipped

\subsubsection{BCPreWindowGap}
\label{BCPreWindowGap}

This is the central flat gap left in the previous windowing operation.
Usually it is set to $0.75 * BCPreWindowLen$, i.e. 768 samples with the
standard BCPreWindowLen value.

\subsubsection{BCNormFactor}
\label{BCNormFactor}

Initial normalization of the input impulse response. 0 means no
normalization, which is the default.

\subsubsection{BCNormType}
\label{BCNormType}

Type of normalization applied. M means max normalization, i.e. the input
signal is rescaled so that the maximum value of the samples is equal to
the normalization factor. E means Euclidean normalization (L2 Norm),
i.e. the input signal is rescaled so that the RMS value of the signal is
equal to the normalization factor. S means sum normalization (L1 Norm),
i.e. the input signal is rescaled so that the sum of the absolute values
of the samples is equal to the normalization factor. P means peak
normalization i.e. the input signal is rescaled so that the highest peak
in the signal magnitude response is equal to the normalization factor.

\subsubsection{BCDLType, BCDLMinGain, BCDLStartFreq, BCDLEndFreq, BCDLStart,
BCDLMultExponent}
\label{BCDipLimiting}

These parameters are used to set a mild dip limiting on the input impulse
response. For a detailed description of these parameters see the similar
procedure described in section \ref{DLDipLimiting}. This stage is used
just to prevent overflow or underflow problems in subsequent stages so
under standard conditions there is no need at all to change these
parameters.

\subsection{MC - Microphone Compensation}
\label{MicrophoneCompensation}

Within this stage the microphone transfer function is invertend and
applied to the input impulse response to compensate for any microphone
aberration. If you want a microphone compensated filter you have to
enable this stage.

The inversion is carried out by direct inversion of the values
supplied in the microphone compensation file. So it is assumed that
the microphone response is easily invertible. This is usually true
with any decent microphone.

\subsubsection{MCFilterType}
\label{MCFilterType}

This is the type of filter used for the microphone compensation stage
stage. N means that the mic compensation stage is disabled, L means linear
phase filtering, M means minimum phase filtering.

\subsubsection{MCInterpolationType}
\label{MCInterpolationType}

This parameter  is the same  as the  PSInterpolationType  parameter (see
section \ref{PSInterpolationType})  but applied  to the mic compensation
filter. The default is H.

\subsubsection{MCMultExponent}
\label{MCMultExponent}

The multiplier exponent  used for the homomorphic  deconvolution used to
compute the minimum phase compensation filter. Usually a value of 2 or 3
is used.

\subsubsection{MCFilterLen}
\label{MCFilterLen}

Length of the FIR  filter used for microphone compensation. Usually
between 16384 and 65536.

\subsubsection{MCNumPoints}
\label{MCNumPoints}

Number of points  used for the  definitions of the  microphone frequency
response. If this parameter is 0 DRC  automatically counts the number of
lines  in the   microphone   frequency   response  file.  See  the following
parameters   for  details  about  the   microphone  frequency   response
compensation.

\subsubsection{MCPointsFile}
\label{MCPointsFile}

This is the name of the file which contains the microphone frequency
response to be compensated. The file format is identical to the one
defined for the target frequency response (see section
\ref{PSPointsFile}). Again any phase specification get wiped out if
minimum phase filtering is used. This usually isn't a problem because
most microphones suited for measurement tasks are minimum phase systems,
so the minimum phase compensation filter already has exactly the phase
response needed to compensate for the microphone phase response.

In the sample directory there's a sample compensation file (wm-61a.txt)
which is a generic compensation file for the Panasonic WM-61A electrect
capsule. This file has been derived from average values available on the
Internet, so don't expect to get perfect linear frequency response using
it. There could be some difference among different capsules of the same
type. In the same directory there's also a compensation file for the
Behringer ECM8000 instrumentation microphone. This is the measured
frequency response of a single unit, i.e. it isn't even derived from an
average over many samples, so it may be even less reliable than the
WM-61A compensation file.

\subsubsection{MCMagType}
\label{MCMagType}

This  parameter  selects  how  the  amplitude  of the  target  frequency
response is  defined. L  means linear  amplitude  ($0.5$ means  half the
level, i.e about  $-6$ dB), D means  that the amplitude  is expressed in
dB.

\subsubsection{MCFilterFile}
\label{MCFilterFile}

This parameter set the file where the impulse response of the microphone
compensation filter will be saved. This might be useful to take a look
at the microphone compensation filter or to use it into some other
program. By default it is disabled, i.e. commented out.

\subsubsection{MCOutWindow}
\label{MCOutWindow}

Final window after  microphone compensation. Default value set to 0, i.e.
no windowing applied.

\subsubsection{MCNormFactor}
\label{MCNormFactor}

Normalization factor for the microphone compensated impulse response.
Usually $0.0$, i.e. disabled.

\subsubsection{MCNormType}
\label{MCNormType}

Normalization type for the microphone compensated impulse response.
Usually E.

\subsubsection{MCOutFile}
\label{MCOutFile}

Output file for the microphone compensated impulse response. Disabled,
i.e. commented out, by default. The file generated by enabling this
parameter might be used as input for the ``createdrcplots'' Octave
script to generate the uncorrected response graph using a microphone
compensated uncorrected response (see section \ref{SampleResults} for
more details).

\subsubsection{MCOutFileType}
\label{MCOutFileType}

Output file type for the microphone compensated impulse response. D =
Double, F = Float, I = Integer.

\subsection{HD - Homomorphic Deconvolution}

\subsubsection{HDMultExponent}
\label{HDMultExponent}

Exponent  of  the   multiplier  of the  FFT  size  used to  perform  the
homomorphic deconvolution. The FFT size used is equal to the first power
of  two    greater    than   or   equal  to  $    BCInitWindow   *  (2 ^
{HDMultExponent})$. Higher  exponents give more  accurate deconvolution,
providing  less circular  convolution  artifacts.

With older DRC  versions achieving  low circular  artifacts was not so
important because they  were masked by the higher  pre-echo artifacts in
other steps.  Starting  with  version  2.0.0 it is  possible  to achieve
really low pre echo  artifacts so  circular artifacts  now are an issue,
because when truncated  by the pre-echo  truncation  inversion procedure
they may cause errors on the phase correction. In this situation a value
of at least 3 is suggested.

\subsubsection{HDMPNormFactor}
\label{HDMPNormFactor}

Normalization factor for the minimum phase component. Usually 1.

\subsubsection{HDMPNormType}
\label{HDMPNormType}

Normalization type for the minimum phase component. Usually E.

\subsubsection{HDMPOutFile}
\label{HDMPOutFile}

Output file for the minimum phase component. Usually not used (commented
out).

\subsubsection{HDMPOutFileType}
\label{HDMPOutFileType}

Output file type for the minimum phase component. D = Double, F = Float,
I = Integer.

\subsubsection{HDEPNormFactor}
\label{HDEPNormFactor}

Normalization factor for the excess phase component. Usually 1.

\subsubsection{HDEPNormType}
\label{HDEPNormType}

Normalization type for the excess phase component. Usually E.

\subsubsection{HDEPOutFile}
\label{HDEPOutFile}

Output file for the excess phase  component. Usually not used (commented
out).

\subsubsection{HDEPOutFileType}
\label{HDEPOutFileType}

Output file type for the excess phase  component. D = Double, F = Float,
I = Integer.

\subsection{MP - Minimum phase Prefiltering}

\subsubsection{MPPrefilterType}
\label{MPPrefilterType}

This parameter can be either B for the usual band windowing prefiltering
stage or S for the sliding lowpass method. The first method splits the
input response into log spaced bands and window them depending on some
parameters but basically with a window length which decrease
exponentially with the center frequency of the band. The sliding lowpass
method instead filters the impulse response with a time varying lowpass
filter with a cutoff frequency which decreases exponentially with the
sample position with respect to the time axis zero. This last one is a
stepless procedure.

Using either  a  lowercase b  or s for  the  MPPrefilterType  parameters
enable the  single  side version  of the  prefiltering   procedures. The
procedure is applied starting from the impulse center, leaving the first
half of  the  impulse  response   unchanged.  This gives  less  pre-echo
artifacts, and  should be  used when the  pre-echo truncation  inversion
procedure is used. Please remember to set the prefiltering parameters to
values which are adequate for the procedure used.

\subsubsection{MPPrefilterFctn}
\label{MPPrefilterFctn}

This parameter sets the type of  prefiltering function  used, i.e. P for
the usual inverse proportional function, or B for the bilinear transform
based prefiltering function. For a comparison between the two functions
see  figure   \ref{BPComparison}. The default is B.

\subsubsection{MPWindowGap}
\label{MPWindowGap}

This parameter changes a little the window function (Blackman) used for
the band windowing prefiltering stage. It sets a small flat unitary gap,
whose length is expressed in samples, at the center of the window
function, so that even if the impulse center is slightly misaligned with
respect to the time axis zero there is no high frequency overcorrection.
For band windowing prefiltering procedure usually this overcorrection is
in the order of $0.1 - 0.2$ dB at 20 kHz for errors of 2 to 3 samples,
so it is not important at all in real world situations, but if you want
to fix even this small problem this parameter lets you do it.

MPWindowGap should never  be more than 2 sample  less than MPUpperWindow
and it  is  usually  no more  than  few  samples  (5 to  10). If  in any
situation it  is bigger  than the  calculated  window DRC  automatically
reduces the gap to 2 less than the applied window. When MPWindowGap is 0
DRC behaves exactly as in the older  versions. For the sliding lowpass
procedure this sets just  the window gap used for  the initial windowing
before the procedure starts.

\subsubsection{MPLowerWindow (*)}
\label{MPLowerWindow}

Length of the window for the minimum phase component prefiltering at the
bottom end of the  frequency range.  Longer windows  cause DRC to try to
correct  a  longer  part of  the  impulse  response  but  cause  greater
sensibility to the listening position.  Typical values are between 16384
and 65536. MPLowerWindow must not be greater than BCInitWindow.

\subsubsection{MPUpperWindow (*)}
\label{MPUpperWindow}

Length of the window for the minimum phase component prefiltering at the
upper end of  the frequency  range. Longer  windows cause  DRC to try to
correct  a  longer  part of  the  impulse  response  but  cause  greater
sensibility on the listening position. Typical values are between 22 and
128. MPUpperWindow must  be not greater than  MPLowerWindow, and usually
is much shorter than that.

\subsubsection{MPStartFreq}
\label{MPStartFreq}

Start  frequency  for the  prefiltering   stage. Usually  20 Hz  or just
something less.

\subsubsection{MPEndFreq}
\label{MPEndFreq}

End frequency for the prefiltering stage. Usually set at 20
kHz, i.e. 20000. Of course you  must be using a sample rate which
is greater than 40 kHz to set this above 20 kHz.

\subsubsection{MPWindowExponent (*)}
\label{MPWindowExponent}

This is the  exponent  used in  the  frequency  dependent  window length
computation for the band  windowing procedure,  or in the computation of
the time dependent cutoff frequency for the sliding lowpass procedure.

The window  length for  band windowing  is  computed with the  following
expression:

\begin{displaymath}
W = \frac{1}{A * (F + Q) ^ {WE}}
\end{displaymath}
Where W is the window  length, F is the normalized  frequency, WE is the
window   exponent,   A  and  Q are   computed  so  that  W is  equal  to
MPLowerWindow at MPStartFreq and is equal to MPUpperWindow at MPEndFreq.
If you set MPLowerWindow equal to the value used for MPInitWindow in DRC
1.2, set  MPWindowExponent  to  the same  value of  version  1.2 and set
MPUpperWindow to the value  you got at the upper  limit of the frequency
range in version 1.2 you should get  results much similar to the 1.2 DRC
release.

In a  similar   way the   cutoff  frequency   for the  sliding   lowpass
prefiltering stage is computed with:

\begin{displaymath}
F = \frac{1}{A * (W + Q) ^ {WE}}
\end{displaymath}
with identical  parameters  but  reversed  perspective, i.e.  the cutoff
frequency  is computed  from  the window  length and  not the  other way
around. In both cases W and F are considered normalized between 0 and 1.

These parametric  functions are used  when the proportional  function is
selected using the  MPPrefilterFctn (see section  \ref{MPPrefilterFctn})
parameter.   The   parametric   functions   derived  from  the  bilinear
transformation are quite  different and more  complicated, so they aren't
explained here.

Changing the window exponent provides different prefiltering curves, see
section \ref{FrequencyDependentWindowing} for a deeper explanation.
Increasing the window exponent gives higher correction in the midrange.
Typical values are between $0.7$ and $1.2$.

\subsubsection{MPFilterLen}
\label{MPFilterLen}

Filter  length,  in taps,  used to  perform  band  splitting or  sliding
lowpass  prefiltering of the  input signal.  Higher values  give better
filter resolution but  require a longer computation.  Typical values for
band windowing are  between 4096 and  65536. Sometimes  may be useful to
use short filters (64 - 512 taps) to  get a more ``fuzzy'' correction at
lower frequencies.

With the sliding lowpass procedure  similar filters should be used.
Usually the filter length is in the 512 - 65536 range. Short filters (16
- 64 taps) gives a similar fuzzy  correction at the bottom end, but with
a different behaviour than band windowing.

\subsubsection{MPFSharpness (*)}
\label{MPFSharpness}

This  parameter   applies  only to  the  sliding  lowpass   prefiltering
procedure and control  the sharpness  of the filtering  performed in the
filtered region of the time-frequency plane. A value of 1.0 provides the
same  behaviour  of  version  2.3.1  of  DRC and  provides  the  maximum
allowable filtering  sharpness without  affecting the  direct sound, but
also creates a  substantial amount  of spectral spreading  in the filter
transition region of the time-frequency plane. Values above 1.0 increase
the spectral spreading up to a point  where it starts affecting also the
direct sound, with the  introduction of some  ripple in the direct sound
itself. Values below 1.0  reduce the spectral  spreading in the filtered
region at the  expense of a  little reduction  in the  filter sharpness.
Typical values for this  parameter are between  $0.1$ and $0.75$, with a
default value of $0.25$.

\subsubsection{MPBandSplit}
\label{MPBandSplit}

Fractional   octave  splitting  of band  windowing.   Band windowing  is
performed in $  1 / {MPBandSplit}  $ of  octave bands.  Usual values are
between 2 and 6. The higher this value the higher should be MPFilterLen.
Values greater than 6 usually give no improvements.

For the sliding lowpass  prefiltering this just  gives the rate at which
log messages are reported  during the prefiltering  procedure and has no
effect on the prefiltering procedure itself, which is always stepless.

\subsubsection{MPHDRecover}
\label{MPHDRecover}

After prefiltering the minimum phase  component may be no longer minimum
phase,  with  a bit  of  excess  phase  component  added.  Setting  this
parameter  to  Y  enable a  second   homomorphic  deconvolution   on the
prefiltered minimum phase component to make it minimum phase again. This
is important especially  if the pre-echo truncation  inversion procedure
is used. This  procedure assumes that  the minimum phase  part really is
minimum phase, so  skipping this step  may cause it to  fail in avoiding
pre-echo artifacts.

\subsubsection{MPEPPreserve}
\label{MPEPPreserve}

Setting this to Y causes  the excess phase part  of the filtered impulse
response to be preserved  after the MPHDRecover  step. This excess phase
part is  then  convolved  with the  excess  phase part  of the  filtered
impulse response  to preserve it and  invert it. This  provides a slight
improvement in the direct sound phase response. The default value is Y.

\subsubsection{MPHDMultExponent}
\label{MPHDMultExponent}

Exponent  of  the   multiplier  of the  FFT  size  used to  perform  the
homomorphic deconvolution described above. The FFT size used is equal to
the first power of two greater than or equal to $ MPPFFinalWindow * (2 ^
{MPHDMultExponent})$.  Higher exponents give more  accurate results, but
require a longer computation. Usually a value of 2 or 3 is used. If this
parameter is less than 0 no multiplier  will be used. Be careful because
if the FFT size isn't a power of two  the procedure can take a long time
to complete.

\subsubsection{MPPFFinalWindow}
\label{MPPFFinalWindow}

Final window of the prefiltering stage. Usually the same as MPLowerWindow
or just something more. If set to 0 no windowing is applied.

\subsubsection{MPPFNormFactor}
\label{MPPFNormFactor}

Normalization factor for the minimum phase component after prefiltering.
Usually $0$.

\subsubsection{MPPFNormType}
\label{MPPFNormType}

Normalization  type for  the minimum  phase  component after  windowing.
Usually E.

\subsubsection{MPPFOutFile}
\label{MPPFOutFile}

Output  file for  the minimum  phase  component  after  band  windowing.
Usually not used (commented out).

\subsubsection{MPPFOutFileType}
\label{MPPFOutFileType}

Output file type  for the minimum  phase component after  windowing. D =
Double, F = Float, I = Integer.

\subsection{DL - Dip Limiting}
\label{DLDipLimiting}

\subsubsection{DLType}
\label{DLType}

To prevent numerical instabilities during the inversion stage, deep dips
in the frequency response must be limited or truncated. This parameter
sets the type of dip limiting performed. L means linear phase, i.e. it
applies a linear phase filter that removes dips below a given threshold,
M means minimum phase, i.e. it uses a minimum phase filter to achieve
the same result. Both procedures are also available with a logarithmic
frequency weighting of the magnitude response, so that, for example, the
20 Hz - 200 Hz range weigths the same as the 2000 Hz - 20000 Hz range.
Setting DLType to P activates the log weighted verions of the linear
phase dip limiting and setting DLType to W activates the log weighted
verions of the minimum phase dip limiting.

Starting   with  version  2.0.0  DRC  performs  this  step  only  on the
prefiltered  minimum  phase  part,  just  before  performing the  second
homomorphic deconvolution, if enabled. So if the MPHDRecover parameter is
set to Y and the MPEPPreserve  parameter is set  to N there is almost no
difference    between  the  two  procedures,   because  the   subsequent
homomorphic  deconvolution stage wipes  out any phase  difference giving
just a minimum  phase  signal. Any  difference  would be  caused just by
numerical errors.

\subsubsection{DLMinGain}
\label{DLMinGain}

This is  the minimum  gain  allowed  in the  frequency  response  of the
prefiltered signal.  Values lower  than this will be  truncated. Typical
values are  between  $0.1$ and  $0.5$.  These are  absolute  values with
respect to the RMS value,  i.e. $0.1$ is about  $-20$ dB, $0.5$ is about
$-6$ dB.

\subsubsection{DLStartFreq}
\label{DLStartFreq}

Start frequency where the  reference RMS level  used for dip limiting is
computed.

\subsubsection{DLEndFreq}
\label{DLEndFreq}

End frequency  where the  reference RMS level  used for  dip limiting is
computed.

\subsubsection{DLStart}
\label{DLStart}

Setting this parameter  to a value  between $0.0$ and  $1.0$ enables the
``soft clipping'' dip  limiting procedure.  Everything below $ DLStart *
DLMinGain $, with respect to the RMS value, get rescaled so that it ends
up between $ DLStart * DLMinGain $  and DLMinGain. Values for this
parameter usually are between $0.5$  and $0.95$, with a typical value of
$0.75$. Setting this parameter to a value equal to or greater then $1.0$
cause DRC to switch to hard clipping of the frequency response.

\subsubsection{DLMultExponent}
\label{DLMultExponent}

Exponent of the multiplier of the FFT size used to perform the dip
limiting stage. The FFT size used is equal to the first power of two
greater than or equal to $ ({MPBWFinalWindow} + {EPBWFinalWindow} - 1) *
(2 ^ {DLMultExponent}) $. Higher exponents give more accurate dip
limiting, but requires a longer computation. Usually a value of 2 or 3
is used. If this parameter is less than 0 no multiplier will be used. Be
careful because if the FFT size isn't a power of two the procedure might
take a long time to complete.

\subsection{EP - Excess phase Prefiltering}

The excess phase prefiltering  is performed pretty  much the same way as
the minimum phase  prefiltering, so the parameters  are almost identical,
even though with different values.

\subsubsection{EPPrefilterType}
\label{EPPrefilterType}

Same as MPPrefilterType but for the excess phase component.

\subsubsection{EPPrefilterFctn}
\label{EPPrefilterFctn}

Same as MPPrefilterFctn but for the excess phase component.

\subsubsection{EPWindowGap}
\label{EPWindowGap}

Same as MPWindowGap but for the excess phase component.

\subsubsection{EPLowerWindow (*)}
\label{EPLowerWindow}

Same as MPLowerWindow but for the excess phase component. Typical values
are between 1024 and 4096. As a rule of thumb you can take:

\begin{displaymath}
{EPLowerWindow}  =  {MPLowerWindow} / A
\end{displaymath}
with $A$ going from $16$ to $32$ and a typical value of $24$.
EPLowerWindow must be not greater than BCInitWindow.

\subsubsection{EPUpperWindow (*)}
\label{EPUpperWindow}

Same as MPUpperWindow but for the excess phase component. Typical values
are between 22 and 128.  As a  rule  of thumb  you  can  take:

\begin{displaymath}
EPUpperWindow = MPUpperWindow
\end{displaymath}

\subsubsection{EPStartFreq}
\label{EPStartFreq}

Start  frequency  for the  prefiltering   stage. Usually  20 Hz  or just
something less.

\subsubsection{EPEndFreq}
\label{EPEndFreq}

End frequency for the prefiltering stage. Usually set at 20
kHz, i.e. 20000. Of course you  must be using a sample rate which
is greater than 40 kHz to set this above 20 kHz.

\subsubsection{EPWindowExponent (*)}
\label{EPWindowExponent}

Same as   MPWindowExponent  but  for  the excess  phase  component.  See
discussion  on  MPWindowExponent.  Usual values  for this  parameter are
between $0.5$ and $1.2$, depending  on the value of EPInitWindow. As
a rule of thumb you can take:

\begin{displaymath}
EPWindowExponent = MPWindowExponent
\end{displaymath}

\subsubsection{EPFilterLen}
\label{EPFilterLen}

Filter length,  in taps,  used to  perform band  splitting  of the input
signal or  sliding  lowpass  prefiltering.  Higher  values  gives better
filter resolution but  require a longer computation.  Typical values for
band windowing are  between 4096 and  65536. Sometimes  may be useful to
use short filters (64 - 512 taps) to  get a more ``fuzzy'' correction at
lower frequencies.

With the sliding lowpass procedure  similar filters should be used.
Usually the filter length is in the 512 - 65536 range. Short filters (16
- 64 taps) gives  a similar  fuzzier correction  at the  bottom end, but
with a different behaviour than band windowing.

This value is usually equal to MPFilterLen.

\subsubsection{EPFSharpness (*)}
\label{EPFSharpness}

Same as MPFSharpness but applied to the excess phase part.

\subsubsection{EPBandSplit}
\label{EPBandSplit}

Fractional   octave  splitting  of band  windowing.   Band windowing  is
performed in $  1 / {MPBandSplit}  $ of  octave bands.  Usual values are
between 2 and 6. The higher this value the higher should be MPFilterLen.
Values greater than 6 usually give no improvements.

For the sliding lowpass  prefiltering this just  gives the rate at which
log  messages  are  reported  and  has no  effect  on the   prefiltering
procedure, which is always stepless.

This value is usually equal to MPBandSplit.

\subsubsection{EPPFFinalWindow}
\label{EPPFFinalWindow}

Final window of the prefiltering stage. Usually the same as EPInitWindow
or just something more. If set to 0 no windowing is applied.

\subsubsection{EPPFFlatGain}
\label{EPPFFlatGain}

After  band   windowing   the  excess  phase   component   usually  need
reequalization to get the  flat frequency response  it must have. This
is the gain applied with  respect to the RMS level  of the signal to get
this flat  frequency  response.  Usually 1, a  value of 0  disables this
step. Skipping  this  step, i.e.  setting this  parameter to  0, usually
gives bad results.

\subsubsection{EPPFOGainFactor}
\label{EPPFOGainFactor}

This parameter controls how the excess phase flattening set by the
previous parameter is performed. Setting this to 0 tries to get a
perfectly flat excess phase component, as in version 1.3.0 of DRC. This
parameter has been introduced to balance between the need of a flat
excess phase response and a perfect control of the direct sound, usually
achieved without any flattening. Unfortunately so far the supposed
balance always proved really difficult to find in any real world
situation, so this parameter is always set to 0 in the standard
configuration file. The procedure has been left just for experimental
purposes if some unusual situation need to be handled.

Furthermore this parameter applies only to the linear phase and minimum
phase excess phase flattening, it isn't available for the D type of
excess phase flattening.

\subsubsection{EPPFFlatType}
\label{EPPFFlatType}

This is the  type of procedure  adopted for  the excess  phase component
renormalization. L means applying  linear phase renormalization, M means
applying  minimum  phase   renormalization,  D  means  applying  another
homomorphic   deconvolution   stage to  extract  just  the excess  phase
component of the prefiltered excess  phase component. L applies a linear
phase filter that equalizes the excess phase amplitude response to flat,
M means minimum phase, i.e. it uses a minimum phase filter to achieve the
same result. The D procedure provides the same effect of the M procedure
when EPPFOGainFactor is  equal to $0$. Any  difference is just caused by
numerical errors.

\subsubsection{EPPFFGMultExponent}
\label{EPPFFGMultExponent}

Exponent of the multiplier of the FFT size used to perform the frequency
response flattening.  The FFT size  used is equal to  the first power of
two    greater    than   or  equal   to  $    EPBWFinalWindow    *  (2 ^
{EPPFFGMultExponent})  $. Higher  exponents give more  accurate results,
but require a longer computation. This parameter should be set using the
same criteria  described in  HDMultExponent.  If this  parameter is less
than 0 no multiplier  will be used.  Be careful because  if the FFT size
isn't a power of two the procedure might take a long time to complete.

\subsubsection{EPPFNormFactor}
\label{EPPFNormFactor}

Normalization    factor  for the   excess  phase  component  after  band
windowing.  Usually 0, i.e.  disabled.

\subsubsection{EPPFNormType}
\label{EPPFNormType}

Normalization  type  for the  excess phase  component  after  windowing.
Usually E.

\subsubsection{EPPFOutFile}
\label{EPPFOutFile}

Output file for the excess phase  component after windowing. Usually not
used (commented out).

\subsubsection{EPPFOutFileType}
\label{EPPFOutFileType}

Output file  type for the  excess phase  component after  windowing. D =
Double, F = Float, I = Integer.

\subsection{PC - Prefilter Completion}

The prefilter  completion stage  combines the prefiltered  minimum phase
and excess phase  parts together  again. The impulse  response recovered
after prefilter completion defines the impulse response of the system as
seen by the correction applied by DRC.

\subsubsection{PCOutWindow}
\label{PCOutWindow}

Final window after prefiltering completion stage and before impulse
inversion. This is usually between 8192 and 65536. Values greater than
65536 make no sense, giving a filter resolution lower than 1 Hz at a
44.1 kHz sample rate. Furthermore inversion of signals longer than 65536
samples may require a lot of time. Starting with version 2.0.0 this step
is no longer needed with the pre-echo truncation fast deconvolution
inversion method, which works directly on the minimum and excess phase
components from the prefiltering stages. So if PCOutFile is not defined
and ISType is set to T this step is completely skipped.

\subsubsection{PCNormFactor}
\label{PCNormFactor}

Normalization   factor  for the   prefiltered  signal.  Usually  0, i.e.
disabled.

\subsubsection{PCNormType}
\label{PCNormType}

Normalization type for the prefiltered signal. Usually E.

\subsubsection{PCOutFile}
\label{PCOutFile}

Output file  for the  prefiltered  signal. Usually  not used  (commented
out).

\subsubsection{PCOutFileType}
\label{PCOutFileType}

Output file type for the prefiltered  signal. D = Double, F = Float, I =
Integer.

\subsection{IS - Inversion Stage}

\subsubsection{ISType}
\label{ISType}

Type of   inversion  stage.  L uses  the  usual  Toeplitz  least  square
inversion, T activates the pre-echo truncation fast deconvolution.

\subsubsection{ISPETType}
\label{ISPETType}

This sets the type of pre echo truncation applied when ISType is T. f
means a fixed pre-echo truncation, s means a time dependent pre-echo
truncation applied using the usual single side sliding low-pass
procedure, but with reversed behaviour, i.e. only what comes before the
impulse center is processed. Starting with version 2.7.0 this is set to
f and pre-echo truncation is basicly disabled because it is already
carried out by the excess phase prefiltering procedure.

\subsubsection{ISPrefilterFctn}
\label{ISPrefilterFctn}

Same as MPPrefilterFctn but for the pre-echo truncation windowing. It is
used only when ISPETType is set to s.

\subsubsection{ISPELowerWindow}
\label{ISPELowerWindow}

When ISPETType is f this is the number of samples before the impulse
center where the inverted impulse response is considered pre-echo.
Starting with version 2.7.0 this is is usually set to half the value of
EPLowerWindow so that the pre-echo truncation procedure provides just a
mild windowing. When ISPETType is s this is the number of samples
considered pre-echo at the ISPEStartFreq frequency, with a typical value
equal to EPLowerWindow.

\subsubsection{ISPEUpperWindow}
\label{ISPEUpperWindow}

When ISPETType is f this is the number of samples before the impulse
center where the pre-echo region, defined by the previous parameter,
ends, and the full impulse response of the inverted filter should start.
Starting with version 2.7.0 this is usually set to about $0.75 *
ISPELowerWindow$ so that the pre-echo truncation procedure is limited to
a mild windowing used only to avoid small steps in the impulse response
attack caused by small numerical errors. When ISPETType is s this is the
number of sample considered pre-echo at the ISPEEndFreq frequency, with
a typical value equal to EPUpperWindow.

\subsubsection{ISPEStartFreq}
\label{ISPEStartFreq}

Start frequency for the sliding low  pass pre-echo truncation procedure.
Usually 20 Hz. Used only when ISPETType is s.

\subsubsection{ISPEEndFreq}
\label{ISPEEndFreq}

End frequency for  the sliding low  pass pre-echo  truncation procedure.
Usually 20000 Hz. Used only when ISPETType is s.

\subsubsection{ISPEFilterLen}
\label{ISPEFilterLen}

Length of the filter  used for the  pre-echo truncation  sliding lowpass
procedure. Usually 8192. Used only when ISPETType is s.

\subsubsection{ISPEFSharpness}
\label{ISPEFSharpness}

Same as MPFSharpness, but applied to the inversion stage pre-echo
truncation. Here slightly bigger values usually provide better results
because of the shorter windowing. Used only when ISPETType is s. The
default value is $0.5$.

\subsubsection{ISPEBandSplit}
\label{ISPEBandSplit}

For the sliding lowpass  prefiltering this just  gives the rate at which
log  messages  are  reported  and  has no  effect  on the   prefiltering
procedure, which is always stepless. Used only when ISPETType is s.

\subsubsection{ISPEWindowExponent}
\label{ISPEWindowExponent}

Window  exponent  applied to  the pre-echo  truncation  sliding  lowpass
procedure. Usual values  goes from $0.5$ to $1.5$,  with a typical value
of $1.0$. Used only when ISPETType is s.

\subsubsection{ISPEOGainFactor}
\label{ISPEOGainFactor}

This parameter has the  same effect of the  EPPFOGainFactor (see section
\ref{EPPFOGainFactor})  but applied to the  renormalization of the excess
phase part  of the inverse  filter  after pre-echo  truncation.  Used in
conjunction with  the EPPFOGainFactor  parameter, this  parameter can be
used to balance the  amount of  correction applied to  the direct sound
compared to the amount of correction applied to the reverberant field. A
negative value disables the renormalization. Default is $0.0$.

\subsubsection{ISSMPMultExponent}
\label{ISSMPMultExponent}

This is the exponent of the multiplier  for the S inversion stage, using
the longest of the input  and output signals as  a basis. This parameter
should be set  using the same  criterion  used for the  MPHDMultExponent
parameters and a values of at least 3 is suggested.

\subsubsection{ISOutWindow}
\label{ISOutWindow}

Final window after inversion stage. Usually 0, i.e. disabled, with the L
type inversion stage. With the S type this is the output filter size and
can be any length but usually is between 8192 and 65536. If it is 0 than
a length equal to  $ {MPPFFinalWindow}  + {EPPFFinalWindow}  - 1 $, i.e.
the length of the convolution of the two components together, is assumed
and no windowing is applied to the output filter.

\subsubsection{ISNormFactor}
\label{ISNormFactor}

Normalization factor for the inverted signal. Usually 0, i.e. disabled.

\subsubsection{ISNormType}
\label{ISNormType}

Normalization type for the inverted signal. Usually E.

\subsubsection{ISOutFile}
\label{ISOutFile}

Output file for the inverted signal. Usually not used (commented out).

\subsubsection{ISOutFileType}
\label{ISOutFileType}

Output  file  type for the inverted  signal.  D =  Double,  F =  Float, I =
Integer.

\subsection{PT - Psychoacoustic Target}
\label{PTPsychoacousticTarget}

This stage computes a psychoacoustic target response based on the
magnitude response envelope.

\subsubsection{PTType}
\label{PTType}

Defines the type of psychoacoustic target filter to use. N means no
filter, thus skipping the psychoacoustic target stage completely. M
means that a minimum phase filter is used and L means that a linear
phase filter is used. The default is M.

\subsubsection{PTReferenceWindow (*)}
\label{PTReferenceWindow}

This parameter define the size used to window the corrected impulse
response. The windowed response is then used to compute the magnitude
response envelope that the target response is based upon. Usually a
portion of the impulse response going from 150 ms to 500 ms is used. The
default value is 26460, corresponding to a symmetric window, 300 ms long
on each side, at 44100 Hz sample rate.

\subsubsection{PTDLType, PTDLMinGain, PTDLStartFreq, PTDLEndFreq, PTDLStart,
PTDLMultExponent}
\label{PTDipLimiting}

These parameters are used to set a small dip limiting on the corrected
impulse response in order to avoid numerical problems in the inversion
of the magnitude response envelope. For a detailed description of these
parameters see the similar procedure described in section
\ref{DLDipLimiting}. This stage is used just to prevent overflow or
underflow problems so under standard conditions there is no need at all
to change these parameters.

\subsubsection{PTBandWidth (*)}
\label{PTBandWidth}

This parameter define the resolution used for the computation of the
magnitude response envelope. It is defined as fraction of octaves, so a
value of 0.25 means a resolution of 1/4 of octave. Values below 0 down
to -1 causes the adoption of the Bark scale, values below -1 causes the
adoption of the ERB scale. The default value is -2, which means that the
computation is performed on the ERB scale.

\subsubsection{PTPeakDetectionStrength  (*)}
\label{PTPeakDetectionStrength}

This parameter define how close the magnitude response envelope will be
to to the peaks in the unsmoothed spectrum. Higher values provide a
closer match. Typical values are between 5 and 30, with the default
value, based on documented psychoacoustic assumptions, set to 15. Values
above 50 are probably going to cause numerical problems and should be
avoided.

\subsubsection{PTMultExponent}
\label{PTMultExponent}

Multiplier exponent for the computation of the magnitude response
envelope. Default is 0.

\subsubsection{PTFilterLen}
\label{PTFilterLen}

Length of the psychoacoustic target filter. Default set to 65536.

\subsubsection{PTFilterFile}
\label{PTFilterFile}

Output file for the psychoacoustic target filter. Usually not
used (commented out).

\subsubsection{PTFilterFileType}
\label{PTFilterFileType}

Output file type for the psychoacoustic target filter. D = Double, F =
Float, I = Integer.

\subsubsection{PTNormFactor}
\label{PTNormFactor}

Normalization factor for the inverted signal after convolution with the
psychoacoustic target filter. Usually 0, i.e. disabled.

\subsubsection{PTNormType}
\label{PTNormType}

Normalization type for the inverted signal after convolution with the
psychoacoustic target filter. Usually E.

\subsubsection{PTOutFile}
\label{PTOutFile}

Output file for the inverted signal after convolution with the
psychoacoustic target filter. Usually not used (commented out).

\subsubsection{PTOutFileType}
\label{PTOutFileType}

Output file type for the inverted signal after convolution with the
psychoacoustic target filter. D = Double, F = Float, I = Integer.

\subsubsection{PTOutWindow}
\label{PTOutWindow}

Normalization factor for the inverted signal after convolution with the
psychoacoustic target filter. Usually 0, i.e. disabled.

\subsection{PL - Peak Limiting}

The peak limiting stage limits the maximum allowed gain of the filter to
prevent amplification and speaker overload.

\subsubsection{PLType}
\label{PLType}

Type of peak limiting applied. L means linear phase, M means minimum
phase, P means log weighted linear phase, W means log weighted minimum
phase. For an explanation of the log weighted versions of the procedures
see section \ref{DLType}. If PSFilterType is set to T PLType should be
set to M or W to ensure that the initial zero valued part is preserved.
Since version 3.1.2 the default and suggested value for this parameter
is W.

\subsubsection{PLMaxGain}
\label{PLMaxGain}

Maximum gain allowed in  the correction filter.  Peaks in the correction
filter amplitude response greater  than this value will be compressed to
PLMaxGain.  Typical  values are  between 1.2 and  4. These  are absolute
values with respect to the  RMS value, i.e. 1.2  is about 1.6 dB and 4 is
about 12 dB.  This peak  limiting  stage is used  to prevent  speaker or
amplifier overloading, resulting in  dynamic range limitations which are
subjectively worse  than some narrow  dip in the  frequency response. A
typical value is 2.0, i.e. 6 dB.

\subsubsection{PLStart}
\label{PLStart}

Setting this parameter  to a value  between $0.0$ and  $1.0$ enables the
``soft clipping'' peak limiting  procedure. Everything above $ {PLStart}
* {PLMaxGain} $, with respect to the  RMS value, get rescaled so that it
ends up between $ {PLStart} * {PLMaxGain}  $ and about PLMaxGain. Values
for this parameter usually are between  $0.5$ and $0.95$, with a typical
value of $0.75$. Setting  this parameter to a  value equal to or greater
than $1.0$ switch to hard clipping of the magnitude response.

\subsubsection{PLStartFreq}
\label{PLStartFreq}

Start frequency where the reference  RMS level used for peak limiting is
computed. Default value set to 100 Hz. This value should be set typically 
to something like an octave above the minimum reproducible frequency
of your speakers.

\subsubsection{PLEndFreq}
\label{PLEndFreq}

End frequency where  the reference  RMS level used for  peak limiting is
computed. Default value set to 10000 Hz. This value should be set typically 
to something like an octave below the maximum reproducible frequency
of your speakers.

\subsubsection{PLMultExponent}
\label{PLMultExponent}

Exponent of  the multiplier  of the  FFT size  used to perform  the peak
limiting stage.  The FFT size  used is equal  to the first  power of two
greater than  or equal to $  {PSOutWindow}  * (2 ^  {PLMultExponent}) $.
Higher exponents give more accurate peak limiting, but requires a longer
computation.  Usually a value  of 2 or 3 is  used. If this  parameter is
less than 0 no  multiplier will be  used. Be careful  because if the FFT
size  isn't a  power  of two  the  procedure  can  take a  long  time to
complete.

\subsubsection{PLOutWindow}
\label{PLOutWindow}

Final window after peak limiting. Usually 0, i.e. disabled.

\subsubsection{PLNormFactor}
\label{PLNormFactor}

Normalization factor for the final filter. Usually 0, i.e. disabled.

\subsubsection{PLNormType}
\label{PLNormType}

Normalization  type for  the peak limited filter, usually E.

\subsubsection{PLOutFile}
\label{PLOutFile}

Output file for  the peak  limited filter.  Usually disabled  (commented
out).

\subsubsection{PLOutFileType}
\label{PLOutFileType}

Output  file type  for the  final  filter. D  = Double,  F =  Float, I =
Integer.

\subsection{RT - Ringing Truncation}
\label{RingingTruncation}

The ringing  truncation  stage  applies  a further  frequency  dependent
windowing to the correction filter. The truncation parameters are pretty
similar to those  of the prefiltering  stage and usually  have also much
similar values.

\subsubsection{RTType}
\label{RTType}

This parameter can be either B or b  for the band windowing method, S or
s for the sliding lowpass method or  N to disable the ringing truncation
stage.   See    section       \ref{FrequencyDependentWindowing}      and
\ref{MPPrefilterType} for further details.

\subsubsection{RTPrefilterFctn}
\label{RTPrefilterFctn}

Same as MPPrefilterFctn but for the ringing truncation windowing.

\subsubsection{RTWindowGap}
\label{RTWindowGap}

This parameter changes a little the window function (Blackman) used for
the band  windowing  or  the  sliding  lowpass  windowing.  See  section
\ref{MPWindowGap} for further details.

\subsubsection{RTLowerWindow (*)}
\label{RTLowerWindow}

Length of the window at  the bottom end of the  frequency range. Usually
set to the same value of MPLowerWindow.

\subsubsection{RTUpperWindow (*)}
\label{RTUpperWindow}

Length of the window  at the upper  end of the frequency  range. Usually
set to the same value of MPUpperWindow.

\subsubsection{RTStartFreq}
\label{RTStartFreq}

Start  frequency  for the  windowing. Usually  20 Hz  or just
something less.

\subsubsection{RTEndFreq}
\label{RTEndFreq}

End frequency for the windowing. Usually set to 20000.

\subsubsection{RTWindowExponent (*)}
\label{RTWindowExponent}

This is the  exponent  used in  the  frequency  dependent  window length
computation for the band  windowing procedure,  or in the computation of
the time dependent cutoff  frequency for the  sliding lowpass procedure.
See section \ref{MPWindowExponent} for further details. Usually set to the
same value of MPWindowExponent.

\subsubsection{RTFilterLen}
\label{RTFilterLen}

Filter  length,  in taps,  used to  perform  band  splitting or  sliding
lowpass prefiltering  of the input  signal. Usually the  same as the one
used in the prefiltering stage.

\subsubsection{RTFSharpness (*)}
\label{RTFSharpness}

This  parameter   applies  only to  the  sliding  lowpass   prefiltering
procedure and control  the sharpness  of the filtering  performed in the
filtered     region   of  the    time-frequency    plane.  See   section
\ref{MPFSharpness} for further details.

\subsubsection{RTBandSplit}
\label{RTBandSplit}

Fractional     octave   splitting   of  band   windowing.  See   section
\ref{MPBandSplit} for further details.

\subsubsection{RTOutWindow}
\label{RTOutWindow}

Final window of the stage. Usually set to 0, i.e. disabled.

\subsubsection{RTNormFactor}
\label{RTNormFactor}

Normalization factor  for the minimum  phase component  after windowing.
Usually $0$.

\subsubsection{RTNormType}
\label{RTNormType}

Normalization  type for  the minimum  phase  component after  windowing.
Usually E.

\subsubsection{RTOutFile}
\label{RTOutFile}

Output file for the filter after  windowing. Usually not used (commented
out).

\subsubsection{RTOutFileType}
\label{RTOutFileType}

Output file type for the filter after  windowing. D = Double, F = Float,
I = Integer.

\subsection{PS - Postfiltering Stage}
\label{PostfilteringStage}

During the  postfiltering  stage the final  target transfer  function is
applied to the  filter and the filter  is normalized  to suitable values
for the convolver used.

\subsubsection{PSFilterType}
\label{PSFilterType}

This is the type of filter used for the postfiltering stage. L means the
usual linear phase  filtering, M means minimum  phase filtering, T means
minimum phase  filtering with initial  zero truncation.  If the pre-echo
truncation  inversion  is used  and the  final post  filtering  stage is
minimum phase all the filter taps before ISPELowerWindow are zero (there
could be some  roundoff errors that  make them different  from zero, but
considering  them  zero  makes no  difference  for our  needs).  So this
initial all zero  part can be windowed  out without  changing the filter
behaviour. This way  the filter becomes  almost zero  delay, providing a
delay of just ISPELowerWindow samples.  This sometimes may be low enough
to make it usable even with home  theater systems where audio delay is a
major issue.  Of  course to  ensure that  the initial  all  zero part is
preserved the minimum phase peak limiting should also be used.

\subsubsection{PSInterpolationType}
\label{PSInterpolationType}

This parameter defines the type of interpolation used between the points
of the target transfer function. L means the usual linear interpolation,
G means logarithmic interpolation, i.e. interpolation performed on a
bilogarithmic scale, R means interpolation using Uniform Cubic B
Splines, S means interpolation using Uniform Cubic B Splines on a
bilogarithmic scale, P means interpolation on a linear scale using a
monotone Piecewise Cubic Hermite Interpolating Polynomial (PCHIP), H
means interpolation on a logarithmic scale using PCHIP. The logarithmic
interpolation makes the definition of the target transfer function
easier, without the need to define intermediate points to get the
desired behaviour on a bilogarithmic scale. The default is S.

The B Splines interpolation options allow for the definition of smooth
target transfer functions which provides less ringing. Be careful when
using this option because defining the right control points to get the
desired target transfer function might be tricky. B Splines don't
interpolate the supplied points but are instead tangent to the lines
connecting the control point. If you want sharp corners in the transfer
function just place few close control points near to the desired corner.
Remember that B Splines of the type used are unaffected by control
points which are more than two control points away from any given point
on the curve. Take a look at the supplied examples for some simple
transfer function definition.

The PCHIP procedure provides a monotonic interpolation procedure. The
resulting target is less smooth than the one supplied by B Splines, but
being a true interpolation PCHIP makes the definition of the control
points much easier.

The use of the B Spline or PCHIP interpolation procedures is often
useful also for the definition of the mic compensation transfer
function, especially when only few points are available.

\subsubsection{PSMultExponent}
\label{PSMultExponent}

The multiplier exponent  used for the homomorphic  deconvolution used to
compute the  minimum  phase post  filter. Usually  a value  of 2 or 3 is
used.

\subsubsection{PSFilterLen}
\label{PSFilterLen}

Length of the FIR  filter used during  the postfiltering  stage. Usually
between 16384 and 65536.

\subsubsection{PSNumPoints}
\label{PSNumPoints}

Number of points  used for the  definition of the post  filter frequency
response. If this parameter is 0 DRC  automatically counts the number of
lines in the post filter  definition file. See  the following parameters
for further details about the post filter frequency response.

\subsubsection{PSMagType}
\label{PSMagType}

This  parameter  selects  how  the  amplitude  of the  target  frequency
response is  defined. L  means linear  amplitude  ($0.5$ means  half the
level, i.e about  $-6$ dB), D means  that the amplitude  is expressed in
dB.

\subsubsection{PSPointsFile (*)}
\label{PSPointsFile}

File containing the post filter frequency response definition. This file
should contain  PSNumPoints  lines,  each line  in the form  ``Frequency
Gain'', with the gain  expressed as a linear gain  or in dB depending on
the PSMagType  parameter  value. The following  examples  are in dB. The
first line must have a  frequency equal to 0, the  last line must have a
frequency equal to $ {BCSampleRate} / 2 $. A post filter definition file
must have the following format:

\begin{quote}
\begin{verbatim}
0 -40
18 -20
20 0
20000 0
21000 -40
22050 -100
\end{verbatim}
\end{quote}
This is for a 44.1 kHz sample rate.

The post filter stage is usually used to prevent overcompensation in the
subsonic or ultrasonic range, but may  be used also to change the target
frequency response from linear to a more euphonic one.

Starting  from version  2.0.0  DRC lets  you specify  the phase  for the
target post filter stage. Phase specification should be placed after the
amplitude specification  and should  be expressed in  degrees. Following
the example above:

\begin{quote}
\begin{verbatim}
0 -40 0
18 -20 45
20 0 90
20000 0 180
21000 -40 90
22050 -100 0
\end{verbatim}
\end{quote}
If not specified a value of 0 is assumed. Setting a phase different than
0, i.e.  flat, is  useless  within  normal  HiFi systems  in  almost all
circumstances. Furthermore  the phase specification  is used only if the
PSFilterType  is L, else  any phase  specification  is wiped  out by the
minimum phase filter extraction.

\subsubsection{PSOutWindow}
\label{PSOutWindow}

Final  window after  post  filtering.  This  is also  the length  of the
generated correction  filter. Usual  values are between  8192 and 65536.
Filter with  65536 taps gives  about $0.5$  Hz resolution  at $44.1$ kHz
sample rate, 16384 is  usually enough for most  situation and 8192 gives
somewhat good results  with much less  computing needs  during real time
convolution.

\subsubsection{PSNormFactor}
\label{PSNormFactor}

Normalization  factor  for the  correction  filter.  Usually  $1.0$. See
section  \ref{PreventingClipping}  for some  instructions  on how to set
this parameter.

\subsubsection{PSNormType}
\label{PSNormType}

Normalization  type for the  correction  filter. Usually  E. See section
\ref{PreventingClipping}    for some   instructions  on how to  set this
parameter.

\subsubsection{PSOutFile}
\label{PSOutFile}

Output file for the correction filter.  This file contains the filter to
be used with the convolution engine.

\subsubsection{PSOutFileType}
\label{PSOutFileType}

Output file type for the  correction filter. D =  Double, F = Float, I =
Integer.

\subsection{MS - Minimum phase filter extraction Stage}

The minimum phase extraction  stage creates a  minimum phase filter from
the  correction   filter. A  minimum  phase  filter   corrects  just the
magnitude   response  and  the  minimum  phase  part  of the  phase
response, but it is usually almost  artifacts free and as basically zero
latency.  If microphone   compensation  is enabled  the filter  includes
microphone compensation.

\subsubsection{MSMultExponent}
\label{MSMultExponent}

Exponent of the  multiplier  for the homomorphic  deconvolution  used to
extract a zero delay  minimum phase version of  the correction filter. A
value of 2 or 3 is usually enough.

\subsubsection{MSOutWindow}
\label{MSOutWindow}

Output window  size for  the minimum  phase  filter. Typical  values are
about half of PLOutWindow.

\subsubsection{MSFilterDelay}
\label{MSFilterDelay}

This parameter add an initial delay to the filter, making it possible
to align it with other filters. Usually it is set to the same value assigned to
EPLowerWindow, so that the filter has the same latency of the mixed phase
filter when PSFilterType is set to T. If you want a zero delay filter set this
parameter to 0.

\subsubsection{MSNormFactor}
\label{MSNormFactor}

Normalization     factor  for  the   minimum  phase   filter.  The  same
considerations of section \ref{PreventingClipping} should be applied.

\subsubsection{MSNormType}
\label{MSNormType}

Normalization    type  for  the   minimum  phase  filter.  Usually E.  See
section  \ref{PreventingClipping}  for some  instructions  on how to set
this parameter.

\subsubsection{MSOutFile}
\label{MSOutFile}

Output file name for the minimum phase filter.

\subsubsection{MSOutFileType}
\label{MSOutFileType}

Output file type for the minimum phase filter.

\subsection{TC - Test Convolution}

\subsubsection{TCNormFactor}
\label{TCNormFactor}

Normalization  factor  for the output  of the  final convolution  stage.
Usually $0.0$.

\subsubsection{TCNormType}
\label{TCNormType}

Normalization   type for  the output  of the  final  convolution  stage.
Usually E.

\subsubsection{TCOutFile}
\label{TCOutFile}

Output file for the final test  convolution. If this is not supplied the
test convolution stage is skipped.

\subsubsection{TCOutFileType}
\label{TCOutFileType}

Output type for the file above. D = Double, F = Float, I = Integer.

\section{Acknowledgments}

DRC grew up with  the contribution  of many peoples.  The list is really
long, and there's  some chance that I'm  forgetting someone. By the
way here it is the list, in random order:

\begin{itemize}

\item Thanks  to Prof.  Angelo Farina  and Prof.  John  Mourjopoulos for
their papers released in the public  domain. Many DRC algorithms started
from references and explanations found in those papers.

\item Many  thanks to  Anders Torger  for his  BruteFIR package  and his
suggestions.  Without BruteFIR  DRC would  have been just  a programming
exercise, and I would have never started writing it. Anders also gave me
the idea of the sliding lowpass prefiltering procedure.

\item Many  thanks  to  ``Jaco the  Relentless''  for  his  enthusiastic
support and for all the tests on his own HiFi system.

\item Thanks to  Maurizio Mulas for  sending me the  impulse response of
his room as a testbed for some releases and for all his listening tests.

\item Thanks to  Marco Bagna and Alex  ``Flex'' Okely  for their support
during the DRC development and also  for letting me testing DRC on their
high quality HiFi systems.

\item Thanks  to Michele  Spinolo for his  enthusiastic  support and for
writing some documentation about DRC and its functionalities.

\item Many thanks  to ``Jones Rush''  for all his efforts  understanding
how DRC works and for  writing a good step by  step DRC guide, something
that was really missing.

\item Many thanks to Tom Browne for his suggestions and tests on his own
system and his help in optimizing the DRC performances.

\item Many thanks to Ed  Wildgoose for his  suggestions and tests on his
own system, for  providing the perl  script which glsweep  is based upon
and for setting up the DRC Wiki pages.

\item Thanks to  Ulrich ``Uli''  Brueggemann for providing  some filters
generated with a completely  different approach.  Most of the changes of
version 2.5.0 have been implemented after comparing the DRC filters with
those filters.

\item Thanks to Chris  Birkinshaw for providing  the Jack version of the
automatic measuring script.

\item Thanks  to Gregory  Maxwell  for writing  the excellent  Wikipedia
digital room correction article.

\item Many  thanks to  the ALSA team  for providing  a good  Linux sound
infrastructure and  for helping fixing  some nasty bugs  in the TerraTec
EWX 24/96 driver.

\item  Many  thanks to  the  \TeX,  \LaTeX,  Octave,  GNUPlot and  HeVeA
developers  for  providing  the  invaluable  tools  used to  create this
document.

\end{itemize}
Finally many  thanks to  all the  peoples who  have contributed  to DRC,
sometimes without even knowing it. Most of the ideas used to develop DRC
come from public papers, algorithms  and source code found for free over
the Internet.

\section{Similar software}

There are other software packages providing functionalities similar or
comparable to those of DRC. Here are some examples:

\begin{itemize}

\item Acourate: \url{http://www.acourate.com/}

\item Room Eq Wizard: \url{http://www.hometheatershack.com/roomeq/}

\item Audiolense: \url{http://www.juicehifi.com/}

\end{itemize}

\section{Commercial products}

A complete commercial package, based on DRC for the filter generation
procedure, is available from the small Italian company AVA Italy. I have
no involvement in the development of the product so those interested in
this package should contact AVA Italy directly. Contact informations are
available on the AVA Italy web site:

\begin{quote}
\url{http://www.avaitaly.it/}
\end{quote}

\appendix

\clearpage

\section{Sample results}
\label{SampleResults}

In the following pages some graphs with a comparison between the
corrected and uncorrected system are reported. This is of course just a
sample situation and describes what I achieved in my own system. My
uncorrected system show performance figures which are quite uncommon,
partly because it is placed in an heavily damped listening room but
mostly because it has been tuned to give its best with the correction in
place. So the results for the uncorrected system shouldn't be taken as
an example of typical behaviour. The results of the corrected system
show instead the perfomance levels achievable combining active room
correction with traditional passive room treatment. Depending on the
behaviour of the speakers and the listening room, and on the settings
used for DRC, the results could be quite different.

The results presented have been obtained with the psychoacoustic target
stage disabled. All the graphs presented in these section are based on
traditional objective evaluation of the system transfer function. No
psychoacoustics is involved in the graphs generation procedures, so the
results with the psychoacoustic target cannot be evaluated using this
kind of graphs. On the other hand the proposed graphs clearly show that
the correction is able to closely match the supplied target, so proving
that any psychoacoustic target computation would be closely followed by
the correction. This implies that, if the underlying psychoacoustic
model is correct, the results will be as expected.

All the graphs, except the spectrograms and other 3D plots, follow the
same conventions. The uncorrected system is reported with red lines and
the corrected system is reported with blue lines. The spectrograms and
the 3D plots need a specific colormap for proper visualization, which is
of course the same for both the corrected and uncorrected system, so
they can't follow this simple convention.

All the graphs have been prepared with the Octave files available under
the ``src/doc/octave'' directory of the standard distribution. The
``createdrcplots.m'' file contains a function which creates all the
graphs needed to compare two impulse responses and saves them into
encapsulated postscript files or files in any other format supporte by
Octave. To load the raw pcm files created by DRC you can use the
``loadpcm'' function with some Octave commands like:

\begin{quote}
{\scriptsize
\begin{verbatim}
ru = loadpcm("/pathtopcm/rmc.pcm");
rc = loadpcm("/pathtopcm/rtc.pcm");
\end{verbatim}
}
\end{quote}
The uncorrected response could be obtained by uncommenting the MCOutFile
parameter and setting it properly. If you don't have a microphone
compensation file you can simply use the flat target response to provide
a null compensation and get the needed output file only. The corrected
response could be obtained by uncommenting the TCOutFile parameter and
setting it properly. Then the full sets of graphs will be created with an
Octave command like:

\begin{quote}
{\scriptsize
\begin{verbatim}
createdrcplots(ru,-1,"R Uncorrected",rc,-1,"R Corrected","/pathtographs/","R");
\end{verbatim}
}
\end{quote}

Graph in a different format than the standard encapsulated postscript
may be easily obtained by suppling some further parameters to the
createdrcplots procedure. For example to create the graphs in PNG format
a command like this might be used:

\begin{quote}
{\scriptsize
\begin{verbatim}
createdrcplots(ru,-1,"R Uncorrected",rc,-1,"R Corrected","/pathtographs/","R",".png","-dpng");
\end{verbatim}
}
\end{quote}
All formats supported by the Octave print command may be used. Se the
Octave documentation for further details.

You need at least Octave 3.2.3, along with Octave-Forge packages ``plot''
and ``signal'', and GnuPlot version 4.3 or newer, for these scripts to work. 
The scripts have been tested with octave 3.6.2 and GnuPlot 4.4.4. Beware
that the latest versions of Octave don't use Gnuplot as the default
plotting toolkit so you might need to issue a command like:

\begin{quote}
{\scriptsize
\begin{verbatim}
graphics_toolkit('gnuplot');
\end{verbatim}
}
\end{quote}
to get the scripts working. Octave can be downloaded from:

\begin{quote}
\url{http://www.octave.org/}
\end{quote}
Michele Spinolo prepared a \LaTeX\ document which packages the full set
of graphs into a single file. The \LaTeX\ script is named
``drc-graphs.tex'' and is available under the ``src/doc'' directory. The
script could be used for pdf or postscript file creation, or to create
HTML files using HeVeA, and maybe also wiht Latex2Html. The graphs
should be created using ``T'' as the graphs prefix name in the
``createdrcplots'' function above, else you have to edit the header of
the script to change the graph prefix. HeVeA and Latex2Html are
available at the following sites:

\begin{quote}
\url{http://www.latex2html.org/}\newline
\url{http://pauillac.inria.fr/~maranget/hevea/}
\end{quote}

\subsection{Time response}
\label{SampleResultsTimeResponse}

The first series of graphs show the effect of the correction in the time
domain.  The  correction   provides  a  clear  improvement  in  the time
response, with an effect  that becomes longer and  longer in time as the
frequency decrease, as expected.

\begin{figure}
\begin{center}
\includegraphics[width=\pctwidth\textwidth,keepaspectratio]{figures/SR-L-IRStepResponse}
\includegraphics[width=\pctwidth\textwidth,keepaspectratio]{figures/SR-R-IRStepResponse}
\caption{\label{StepResponseFullRange}   Corrected and  uncorrected step
response comparison. The  corrected step response  is much closer to the
expected  exponential  decay than  the uncorrected  one, at  least up to
above 10 ms.}
\end{center}
\end{figure}

\begin{figure}
\begin{center}
\includegraphics[width=\pctwidth\textwidth,keepaspectratio]{figures/SR-L-IRFullRange}
\includegraphics[width=\pctwidth\textwidth,keepaspectratio]{figures/SR-R-IRFullRange}
\caption{\label{ImpulseResponseFullRange} Corrected and uncorrected
impulse response comparison. The corrected impulse response becomes much
similar to a (minimum phase) bandlimited Dirac spike for about 1 ms.
This implies a close to perfect phase response at least for the early
direct sound.}

\end{center}
\end{figure}

\begin{figure}
\begin{center}
\includegraphics[width=\pctwidth\textwidth,keepaspectratio]{figures/SR-L-IRFullRangeEnvelope}
\includegraphics[width=\pctwidth\textwidth,keepaspectratio]{figures/SR-R-IRFullRangeEnvelope}
\caption{\label{EnvelopeFullRange}   Impulse  response  envelope for the
corrected  and  uncorrected  system.}
\end{center}
\end{figure}

\begin{figure}
\begin{center}
\includegraphics[width=\pctwidth\textwidth,keepaspectratio]{figures/SR-L-IRFullRangeETC}
\includegraphics[width=\pctwidth\textwidth,keepaspectratio]{figures/SR-R-IRFullRangeETC}
\caption{\label{ETCFullRange}   Time-energy  response  (impulse response
envelope plotted with a logarithmic  magnitude scale) for the corrected
and uncorrected system.}
\end{center}
\end{figure}

\begin{figure}
\begin{center}
\includegraphics[width=\pctwidth\textwidth,keepaspectratio]{figures/SR-L-IR2kHzBrickwall}
\includegraphics[width=\pctwidth\textwidth,keepaspectratio]{figures/SR-R-IR2kHzBrickwall}
\caption{\label{ImpulseResponse2kHzBrickwall}  Corrected and uncorrected
impulse response comparison.  The impulse responses  have been brickwall
filtered at 2 kHz to show  the increased effect  up to the midrange.}
\end{center}
\end{figure}

\begin{figure}
\begin{center}
\includegraphics[width=\pctwidth\textwidth,keepaspectratio]{figures/SR-L-IR2kHzBrickwallEnvelope}
\includegraphics[width=\pctwidth\textwidth,keepaspectratio]{figures/SR-R-IR2kHzBrickwallEnvelope}
\caption{\label{Envelope2kHzBrickwall} Impulse response envelope for the
corrected  and  uncorrected  system.  The  impulse  responses  have been
brickwall  filtered  at 2 kHz  to show  the increased  effect  up to the
midrange.}
\end{center}
\end{figure}

\begin{figure}
\begin{center}
\includegraphics[width=\pctwidth\textwidth,keepaspectratio]{figures/SR-L-IR2kHzBrickwallETC}
\includegraphics[width=\pctwidth\textwidth,keepaspectratio]{figures/SR-R-IR2kHzBrickwallETC}
\caption{\label{ETC2kHzBrickwall} Time-energy response (impulse response
envelope plotted with a logarithmic  magnitude scale) for the corrected
and  uncorrected  system.  The impulse   responses  have been  brickwall
filtered at 2 kHz to show  the increased effect  up to the midrange.}
\end{center}
\end{figure}

\begin{figure}
\begin{center}
\includegraphics[width=\pctwidth\textwidth,keepaspectratio]{figures/SR-L-IR200HzBrickwall}
\includegraphics[width=\pctwidth\textwidth,keepaspectratio]{figures/SR-R-IR200HzBrickwall}
\caption{\label{ImpulseResponse200HzBrickwall} Corrected and uncorrected
impulse response comparison.  The impulse responses  have been brickwall
filtered  at  200  Hz to  show  the  further  increased   effect  in the
bassrange.}
\end{center}
\end{figure}

\begin{figure}
\begin{center}
\includegraphics[width=\pctwidth\textwidth,keepaspectratio]{figures/SR-L-IR200HzBrickwallEnvelope}
\includegraphics[width=\pctwidth\textwidth,keepaspectratio]{figures/SR-R-IR200HzBrickwallEnvelope}
\caption{\label{Envelope200HzBrickwall}   Impulse  response envelope for
the corrected and  uncorrected system.  The impulse  responses have been
brickwall filtered at 200 Hz to show the further increased effect in the
bassrange.}
\end{center}
\end{figure}

\begin{figure}
\begin{center}
\includegraphics[width=\pctwidth\textwidth,keepaspectratio]{figures/SR-L-IR200HzBrickwallETC}
\includegraphics[width=\pctwidth\textwidth,keepaspectratio]{figures/SR-R-IR200HzBrickwallETC}
\caption{\label{ETC200HzBrickwall}      Time-energy   response  (impulse
response envelope plotted  with a logarithmic  magnitude scale) for the
corrected  and  uncorrected  system.  The  impulse  responses  have been
brickwall filtered at 200 Hz to show the further increased effect in the
bassrange.}
\end{center}
\end{figure}

\clearpage
\subsection{Frequency response}
\label{SampleResultsFrequencyResponse}

These  series  of  graphs  show the  effect  of  the  correction  on the
frequency response magnitude  for some different  windows applied to the
time response and with different kind of smoothing applied.

\begin{figure}
\begin{center}
\includegraphics[width=\pctwidth\textwidth,keepaspectratio]{figures/SR-L-MRUnsmoothed1ms}
\includegraphics[width=\pctwidth\textwidth,keepaspectratio]{figures/SR-R-MRUnsmoothed1ms}
\caption{\label{MagnitudeUnsmoothed1ms}   Unsmoothed  frequency response
magnitude,  1  ms Blackman   window.  These  graphs show  the  frequency
response of  the early  direct sound.}
\end{center}
\end{figure}

\begin{figure}
\begin{center}
\includegraphics[width=\pctwidth\textwidth,keepaspectratio]{figures/SR-L-MRUnsmoothed5ms}
\includegraphics[width=\pctwidth\textwidth,keepaspectratio]{figures/SR-R-MRUnsmoothed5ms}
\caption{\label{MagnitudeUnsmoothed5ms}   Unsmoothed  frequency response
magnitude,  5  ms Blackman   window.  These  graphs show  the  frequency
response of the  direct sound.}
\end{center}
\end{figure}

\begin{figure}
\begin{center}
\includegraphics[width=\pctwidth\textwidth,keepaspectratio]{figures/SR-L-MRUnsmoothed200ms}
\includegraphics[width=\pctwidth\textwidth,keepaspectratio]{figures/SR-R-MRUnsmoothed200ms}
\caption{\label{MagnitudeUnsmoothed200ms}  Unsmoothed frequency response
magnitude, bass  range, 200  ms Blackman  window. These  graphs show the
frequency  response of  the bass  range over a  200 ms time  window.}
\end{center}
\end{figure}

\begin{figure}
\begin{center}
\includegraphics[width=\pctwidth\textwidth,keepaspectratio]{figures/SR-L-MRFDWSmoothed}
\includegraphics[width=\pctwidth\textwidth,keepaspectratio]{figures/SR-R-MRFDWSmoothed}
\caption{\label{MagnitudeFDWSmoothedNormal} Frequency response magnitude
smoothed using a frequency  dependent windowing  with windowing settings
close to those  used by  the normal.drc  sample  configuration  file.}
\end{center}
\end{figure}

\begin{figure}
\begin{center}
\includegraphics[width=\pctwidth\textwidth,keepaspectratio]{figures/SR-L-MRFDWSmoothed-1-6}
\includegraphics[width=\pctwidth\textwidth,keepaspectratio]{figures/SR-R-MRFDWSmoothed-1-6}
\caption{\label{MagnitudeFDWSmoothed16Oct}  Frequency response magnitude
smoothed using  a frequency  dependent  windowing providing  a frequency
resolution close to 1/6  of octave smoothing.}
\end{center}
\end{figure}

\begin{figure}
\begin{center}
\includegraphics[width=\pctwidth\textwidth,keepaspectratio]{figures/SR-L-MRFDWSmoothed-1-3}
\includegraphics[width=\pctwidth\textwidth,keepaspectratio]{figures/SR-R-MRFDWSmoothed-1-3}
\caption{\label{MagnitudeFDWSmoothed13Oct}  Frequency response magnitude
smoothed using  a frequency  dependent  windowing providing  a frequency
resolution close to 1/3  of octave smoothing.}
\end{center}
\end{figure}

\begin{figure}
\begin{center}
\includegraphics[width=\pctwidth\textwidth,keepaspectratio]{figures/SR-L-MRBarkSmoothed}
\includegraphics[width=\pctwidth\textwidth,keepaspectratio]{figures/SR-R-MRBarkSmoothed}
\caption{\label{MagnitudeBarkSmoothed} Frequency response magnitude
smoothed over the Bark psychoacoustic scale with many different Blackman windows applied.}
\end{center}
\end{figure}

\begin{figure}
\begin{center}
\includegraphics[width=\pctwidth\textwidth,keepaspectratio]{figures/SR-L-MRERBSmoothed}
\includegraphics[width=\pctwidth\textwidth,keepaspectratio]{figures/SR-R-MRERBSmoothed}
\caption{\label{MagnitudeERBSmoothed} Frequency response magnitude
smoothed the ERB psychoacoustic scale with many different Blackman windows applied.}
\end{center}
\end{figure}

\clearpage
\subsection{Phase response}
\label{PhaseResponse}

This section show some phase response graphs. The phase response becomes
basicly  linear at  least for  the direct  sound,  which implies  also a
constant group delay.

\begin{figure}
\begin{center}
\includegraphics[width=\pctwidth\textwidth,keepaspectratio]{figures/SR-L-PRUnsmoothed1ms}
\includegraphics[width=\pctwidth\textwidth,keepaspectratio]{figures/SR-R-PRUnsmoothed1ms}
\caption{\label{PhaseResponse1} Unsmoothed phase response, 1 ms Blackman
window.}
\end{center}
\end{figure}

\begin{figure}
\begin{center}
\includegraphics[width=\pctwidth\textwidth,keepaspectratio]{figures/SR-L-PRUnsmoothed5ms}
\includegraphics[width=\pctwidth\textwidth,keepaspectratio]{figures/SR-R-PRUnsmoothed5ms}
\caption{\label{PhaseResponse5} Unsmoothed phase response, 5 ms Blackman
window.}
\end{center}
\end{figure}

\begin{figure}
\begin{center}
\includegraphics[width=\pctwidth\textwidth,keepaspectratio]{figures/SR-L-PRUnsmoothed200ms}
\includegraphics[width=\pctwidth\textwidth,keepaspectratio]{figures/SR-R-PRUnsmoothed200ms}
\caption{\label{PhaseResponse200}    Unsmoothed  phase response, bass range, 200 ms
Blackman window.}
\end{center}
\end{figure}

\begin{figure}
\begin{center}
\includegraphics[width=\pctwidth\textwidth,keepaspectratio]{figures/SR-L-PRFDWSmoothed}
\includegraphics[width=\pctwidth\textwidth,keepaspectratio]{figures/SR-R-PRFDWSmoothed}
\caption{\label{PhaseFDWSmoothedNormal}  Phase response smoothed using a
frequency  dependent windowing  with windowing  settings  close to those
used by the normal.drc sample configuration file.}
\end{center}
\end{figure}

\begin{figure}
\begin{center}
\includegraphics[width=\pctwidth\textwidth,keepaspectratio]{figures/SR-L-PRFDWSmoothed-1-6}
\includegraphics[width=\pctwidth\textwidth,keepaspectratio]{figures/SR-R-PRFDWSmoothed-1-6}
\caption{\label{PhaseFDWSmoothed16Oct} Phase response magnitude smoothed
using a frequency dependent  windowing providing  a frequency resolution
close to 1/6 of octave smoothing.}
\end{center}
\end{figure}

\begin{figure}
\begin{center}
\includegraphics[width=\pctwidth\textwidth,keepaspectratio]{figures/SR-L-PRFDWSmoothed-1-3}
\includegraphics[width=\pctwidth\textwidth,keepaspectratio]{figures/SR-R-PRFDWSmoothed-1-3}
\caption{\label{PhaseFDWSmoothed13Oct} Phase response magnitude smoothed
using a frequency dependent  windowing providing  a frequency resolution
close to 1/3 of octave smoothing.}
\end{center}
\end{figure}

\clearpage
\subsection{Time-frequency analysis}
\label{TimeFrequencyAnalysis}

In this   section  some  joint   time-frequency   analysis  results  are
presented. Time-frequency  graphs are more difficult  to understand than
the  graphs   presented  so  far,  but  they  provide  also   invaluable
information about  how the system  under test is working.  The human ear
works using a joint time-frequency analysis too, so these graphs provide
a representation  of the  system  behaviour that  is much  closer to our
subjective perception.

Many graphs show  the spectral decay  of the system.  The spectral decay
isn't exactly the same as the cumulative spectral decay (CSD) often used
for loudspeaker analysis, even though  it is strictly related to it. The
spectral  decay is  obtained  using an  oversampled  short-time  Fourier
transform of the impulse response, being careful to use a window that is
long   enough   to   satisfy   the  Gabor    inequality   (see   section
\ref{FrequencyDependentWindowing} and figure \ref{Gabor} for details).

The correct interpretation of the graphs presented in this section would
require a  book by  itself, so  little  words are  spent  describing the
results achieved. With respect to this any comment is welcome.

\clearpage
\begin{figure}
\begin{center}
\includegraphics[width=\pctwidth\textwidth,keepaspectratio]{figures/SR-L-SDHighRange}
\caption{\label{SpectralDecayHRL}   Left  channel, spectral  decay, high
range.  Spectral  decay  from  2 kHz  to 20 kHz  with a  0.5 ms  sliding
Blackman window.}
\end{center}
\end{figure}

\begin{figure}
\begin{center}
\includegraphics[width=\pctwidth\textwidth,keepaspectratio]{figures/SR-R-SDHighRange}
\caption{\label{SpectralDecayHRR}   Right channel,  spectral decay, high
range.  Spectral  decay  from  2 kHz  to 20 kHz  with a  0.5 ms  sliding
Blackman window.}
\end{center}
\end{figure}

\begin{figure}
\begin{center}
\includegraphics[width=\pctwidth\textwidth,keepaspectratio]{figures/SR-L-SFHighRange}
\caption{\label{SpectralFormationHRL}  Left channel, spectral formation,
high range.   Spectral  formation  from 2 kHz  to 20  kHz with  a 0.5 ms
sliding Blackman window.}
\end{center}
\end{figure}

\begin{figure}
\begin{center}
\includegraphics[width=\pctwidth\textwidth,keepaspectratio]{figures/SR-R-SFHighRange}
\caption{\label{SpectralFormationHRR} Right channel, spectral formation,
high range.   Spectral  formation  from 2 kHz  to 20  kHz with  a 0.5 ms
sliding Blackman window.}
\end{center}
\end{figure}

\begin{figure}
\begin{center}
\includegraphics[width=\pctwidth\textwidth,keepaspectratio]{figures/SR-L-SDMidRange}
\caption{\label{SpectralDecayMRL}    Left channel,  spectral  decay, mid
range.  Spectral  decay  from  200 Hz  to 2000  Hz with  a 5 ms  sliding
Blackman window.}
\end{center}
\end{figure}

\begin{figure}
\begin{center}
\includegraphics[width=\pctwidth\textwidth,keepaspectratio]{figures/SR-R-SDMidRange}
\caption{\label{SpectralDecayMRR}   Right  channel, spectral  decay, mid
range.  Spectral  decay  from  200 Hz  to 2000  Hz with  a 5 ms  sliding
Blackman window.}
\end{center}
\end{figure}

\begin{figure}
\begin{center}
\includegraphics[width=\pctwidth\textwidth,keepaspectratio]{figures/SR-L-SFMidRange}
\caption{\label{SpectralFormationMRL}  Left channel, spectral formation,
mid range. Spectral formation from 200 Hz to 2000 Hz with a 5 ms sliding
Blackman window.}
\end{center}
\end{figure}

\begin{figure}
\begin{center}
\includegraphics[width=\pctwidth\textwidth,keepaspectratio]{figures/SR-R-SFMidRange}
\caption{\label{SpectralFormationMRR} Right channel, spectral formation,
mid range. Spectral formation from 200 Hz to 2000 Hz with a 5 ms sliding
Blackman window.}
\end{center}
\end{figure}

\begin{figure}
\begin{center}
\includegraphics[width=\pctwidth\textwidth,keepaspectratio]{figures/SR-L-SDBassRange}
\caption{\label{SpectralDecayBRL}   Left  channel, spectral  decay, bass
range. Spectral decay from 20 Hz to 200 Hz with a 50 ms sliding Blackman
window. }
\end{center}
\end{figure}

\begin{figure}
\begin{center}
\includegraphics[width=\pctwidth\textwidth,keepaspectratio]{figures/SR-R-SDBassRange}
\caption{\label{SpectralDecayBRR}   Right channel,  spectral decay, bass
range. Spectral decay from 20 Hz to 200 Hz with a 50 ms sliding Blackman
window. }
\end{center}
\end{figure}

\begin{figure}
\begin{center}
\includegraphics[width=\pctwidth\textwidth,keepaspectratio]{figures/SR-L-SFBassRange}
\caption{\label{SpectralFormationBRL}  Left channel, spectral formation,
bass range. Spectral formation from 20 Hz to 200 Hz with a 50 ms sliding
Blackman window. }
\end{center}
\end{figure}

\begin{figure}
\begin{center}
\includegraphics[width=\pctwidth\textwidth,keepaspectratio]{figures/SR-R-SFBassRange}
\caption{\label{SpectralFormationBRR} Right channel, spectral formation,
bass range. Spectral formation from 20 Hz to 200 Hz with a 50 ms sliding
Blackman window. }
\end{center}
\end{figure}

\clearpage
\begin{figure}
\begin{center}
\includegraphics[width=\pctwidth\textwidth,keepaspectratio]{figures/SR-L-SDHighRangeW}
\caption{\label{SpectralDecayHRLW}   Left  channel, spectral  decay, high
range.  Spectral  decay  from  2 kHz  to 20 kHz  with a  1.0 ms  sliding
Blackman window.}
\end{center}
\end{figure}

\begin{figure}
\begin{center}
\includegraphics[width=\pctwidth\textwidth,keepaspectratio]{figures/SR-R-SDHighRangeW}
\caption{\label{SpectralDecayHRRW}   Right channel,  spectral decay, high
range.  Spectral  decay  from  2 kHz  to 20 kHz  with a  1.0 ms  sliding
Blackman window.}
\end{center}
\end{figure}

\begin{figure}
\begin{center}
\includegraphics[width=\pctwidth\textwidth,keepaspectratio]{figures/SR-L-SFHighRangeW}
\caption{\label{SpectralFormationHRLW}  Left channel, spectral formation,
high range.   Spectral  formation  from 2 kHz  to 20  kHz with  a 1.0 ms
sliding Blackman window.}
\end{center}
\end{figure}

\begin{figure}
\begin{center}
\includegraphics[width=\pctwidth\textwidth,keepaspectratio]{figures/SR-R-SFHighRangeW}
\caption{\label{SpectralFormationHRRW} Right channel, spectral formation,
high range.   Spectral  formation  from 2 kHz  to 20  kHz with  a 1.0 ms
sliding Blackman window.}
\end{center}
\end{figure}

\begin{figure}
\begin{center}
\includegraphics[width=\pctwidth\textwidth,keepaspectratio]{figures/SR-L-SDMidRangeW}
\caption{\label{SpectralDecayMRLW}    Left channel,  spectral  decay, mid
range.  Spectral  decay  from  200 Hz  to 2000  Hz with  a 10 ms  sliding
Blackman window.}
\end{center}
\end{figure}

\begin{figure}
\begin{center}
\includegraphics[width=\pctwidth\textwidth,keepaspectratio]{figures/SR-R-SDMidRangeW}
\caption{\label{SpectralDecayMRRW}   Right  channel, spectral  decay, mid
range.  Spectral  decay  from  200 Hz  to 2000  Hz with  a 10 ms  sliding
Blackman window.}
\end{center}
\end{figure}

\begin{figure}
\begin{center}
\includegraphics[width=\pctwidth\textwidth,keepaspectratio]{figures/SR-L-SFMidRangeW}
\caption{\label{SpectralFormationMRLW}  Left channel, spectral formation,
mid range. Spectral formation from 200 Hz to 2000 Hz with a 10 ms sliding
Blackman window.}
\end{center}
\end{figure}

\begin{figure}
\begin{center}
\includegraphics[width=\pctwidth\textwidth,keepaspectratio]{figures/SR-R-SFMidRangeW}
\caption{\label{SpectralFormationMRRW} Right channel, spectral formation,
mid range. Spectral formation from 200 Hz to 2000 Hz with a 10 ms sliding
Blackman window.}
\end{center}
\end{figure}

\begin{figure}
\begin{center}
\includegraphics[width=\pctwidth\textwidth,keepaspectratio]{figures/SR-L-SDBassRangeW}
\caption{\label{SpectralDecayBRLW}   Left  channel, spectral  decay, bass
range. Spectral decay from 20 Hz to 200 Hz with a 100 ms sliding Blackman
window. }
\end{center}
\end{figure}

\begin{figure}
\begin{center}
\includegraphics[width=\pctwidth\textwidth,keepaspectratio]{figures/SR-R-SDBassRangeW}
\caption{\label{SpectralDecayBRRW}   Right channel,  spectral decay, bass
range. Spectral decay from 20 Hz to 200 Hz with a 100 ms sliding Blackman
window. }
\end{center}
\end{figure}

\begin{figure}
\begin{center}
\includegraphics[width=\pctwidth\textwidth,keepaspectratio]{figures/SR-L-SFBassRangeW}
\caption{\label{SpectralFormationBRLW}  Left channel, spectral formation,
bass range. Spectral formation from 20 Hz to 200 Hz with a 100 ms sliding
Blackman window. }
\end{center}
\end{figure}

\begin{figure}
\begin{center}
\includegraphics[width=\pctwidth\textwidth,keepaspectratio]{figures/SR-R-SFBassRangeW}
\caption{\label{SpectralFormationBRRW} Right channel, spectral formation,
bass range. Spectral formation from 20 Hz to 200 Hz with a 100 ms sliding
Blackman window. }
\end{center}
\end{figure}

\begin{figure}
\begin{center}
\includegraphics[width=\pctwidth\textwidth,keepaspectratio]{figures/SR-L-Spectrogram01ms}
\caption{\label{HigResSpectrogram01msL} High resolution spectrograms
from -10 ms to 40 ms, 1 ms Blackman window, 90 dB level range, left
channel. The frequency range is from DC to the Nyquist frequency (22050
Hz) on a linear scale. The uncorrected system is on the top and the
corrected system is on the bottom.}
\end{center}
\end{figure}

\begin{figure}
\begin{center}
\includegraphics[width=\pctwidth\textwidth,keepaspectratio]{figures/SR-R-Spectrogram01ms}
\caption{\label{HigResSpectrogram01msR} High resolution spectrograms
from -10 ms to 40 ms, 1 ms Blackman window, 90 dB level range, right
channel. The frequency range is from DC to the Nyquist frequency (22050
Hz) on a linear scale. The uncorrected system is on the top and the
corrected system is on the bottom.}
\end{center}
\end{figure}

\begin{figure}
\begin{center}
\includegraphics[width=\pctwidth\textwidth,keepaspectratio]{figures/SR-L-Spectrogram20ms}
\caption{\label{HigResSpectrogram20msL} High resolution spectrograms
from -100 ms to 400 ms, 20 ms Blackman window, 90 dB level range, left
channel. The frequency range is from DC to the Nyquist frequency (22050
Hz) on a linear scale. The uncorrected system is on the top and the
corrected system is on the bottom.}
\end{center}
\end{figure}

\begin{figure}
\begin{center}
\includegraphics[width=\pctwidth\textwidth,keepaspectratio]{figures/SR-R-Spectrogram20ms}
\caption{\label{HigResSpectrogram20msR} High resolution spectrograms
from -100 ms to 400 ms, 20 ms Blackman window, 90 dB level range, right
channel. The frequency range is from DC to the Nyquist frequency (22050
Hz) on a linear scale. The uncorrected system is on the top and the
corrected system is on the bottom.}
\end{center}
\end{figure}

\clearpage
\subsection{Wavelet cycle-octave analysis}
\label{WaveletAnalysis}

The  wavelet    analysis  is  a  different   method  for   performing  a
time-frequency  analysis or, to be  precise, a time-scale  analysis. For
certain kind of wavelets  the scale axis could  be mapped to a frequency
scale,  allowing for  the usual   time-frequency  interpretation  of the
time-scale plots.

Wavelets have  the  advantage of  being easier  to map to a  logarithmic
frequency  scale. To  further  help the  correct  interpretation  of the
graphs the time scale is also stretched,  depending on the frequency, so
that the  time scale  is expressed  in  cycles of  the sine  wave of the
corresponding   frequency. The  end result  is a  graph that  provides a
tiling of the time-frequency plane  which is visually quite close to the
kind of time-frequency analysis performed by our auditory system.

The graphs are classical cycle-octave scalograms based on the Morlet
wavelet, tuned for different tradeoffs between time and frequency
resolution.

% Morlet cycle-octave scalogram, alta risoluzione

\clearpage
\begin{figure}
\begin{center}
\includegraphics[width=\pctwidth\textwidth,keepaspectratio]{figures/SR-L-MorletScalogramHTRENVSD}
\caption{\label{MorletScalogramHTRENVSDL} Left channel, Morlet
cycle-octave scalogram envelope, high time resolution, spectral decay. }
\end{center}
\end{figure}

\begin{figure}
\begin{center}
\includegraphics[width=\pctwidth\textwidth,keepaspectratio]{figures/SR-R-MorletScalogramHTRENVSD}
\caption{\label{MorletScalogramHTRENVSDR} Right channel, Morlet
cycle-octave scalogram envelope, high time resolution, spectral decay. }
\end{center}
\end{figure}

\begin{figure}
\begin{center}
\includegraphics[width=\pctwidth\textwidth,keepaspectratio]{figures/SR-L-MorletScalogramHTRENVSF}
\caption{\label{MorletScalogramHTRENVSFL} Left channel, Morlet
cycle-octave scalogram envelope, high time resolution, spectral formation. }
\end{center}
\end{figure}

\begin{figure}
\begin{center}
\includegraphics[width=\pctwidth\textwidth,keepaspectratio]{figures/SR-R-MorletScalogramHTRENVSF}
\caption{\label{MorletScalogramHTRENVSFR} Right channel, Morlet
cycle-octave scalogram envelope, high time resolution, spectral formation. }
\end{center}
\end{figure}

\begin{figure}
\begin{center}
\includegraphics[width=\pctwidth\textwidth,keepaspectratio]{figures/SR-L-MorletScalogramHTRENVMap}
\caption{\label{MorletScalogramHTRENVMapL}   Left channel, Morlet
cycle-octave scalogram envelope, high time resolution, colored map. }
\end{center}
\end{figure}

\begin{figure}
\begin{center}
\includegraphics[width=\pctwidth\textwidth,keepaspectratio]{figures/SR-R-MorletScalogramHTRENVMap}
\caption{\label{MorletScalogramHTRENVMapR}   Right channel, Morlet
cycle-octave scalogram envelope, high time resolution, colored map. }
\end{center}
\end{figure}

\begin{figure}
\begin{center}
\includegraphics[width=\pctwidth\textwidth,keepaspectratio]{figures/SR-L-MorletScalogramHTRETCSD}
\caption{\label{MorletScalogramHTRETCSDL} Left channel, Morlet
cycle-octave scalogram ETC, high time resolution, spectral decay. }
\end{center}
\end{figure}

\begin{figure}
\begin{center}
\includegraphics[width=\pctwidth\textwidth,keepaspectratio]{figures/SR-R-MorletScalogramHTRETCSD}
\caption{\label{MorletScalogramHTRETCSDR} Right channel, Morlet
cycle-octave scalogram ETC, high time resolution, spectral decay. }
\end{center}
\end{figure}

\begin{figure}
\begin{center}
\includegraphics[width=\pctwidth\textwidth,keepaspectratio]{figures/SR-L-MorletScalogramHTRETCSF}
\caption{\label{MorletScalogramHTRETCSFL} Left channel, Morlet
cycle-octave scalogram ETC, high time resolution, spectral formation. }
\end{center}
\end{figure}

\begin{figure}
\begin{center}
\includegraphics[width=\pctwidth\textwidth,keepaspectratio]{figures/SR-R-MorletScalogramHTRETCSF}
\caption{\label{MorletScalogramHTRETCSFR} Right channel, Morlet
cycle-octave scalogram ETC, high time resolution, spectral formation. }
\end{center}
\end{figure}

\begin{figure}
\begin{center}
\includegraphics[width=\pctwidth\textwidth,keepaspectratio]{figures/SR-L-MorletScalogramHTRETCMap}
\caption{\label{MorletScalogramHTRETCMapL} Left channel, Morlet
cycle-octave scalogram ETC, high time resolution, colored map. }
\end{center}
\end{figure}

\begin{figure}
\begin{center}
\includegraphics[width=\pctwidth\textwidth,keepaspectratio]{figures/SR-R-MorletScalogramHTRETCMap}
\caption{\label{MorletScalogramHTRETCMapR} Right  channel, Morlet
cycle-octave scalogram ETC, high time resolution, colored map. }
\end{center}
\end{figure}

% Morlet cycle-octave scalogram, media risoluzione

\clearpage
\begin{figure}
\begin{center}
\includegraphics[width=\pctwidth\textwidth,keepaspectratio]{figures/SR-L-MorletScalogramMTRENVSD}
\caption{\label{MorletScalogramMTRENVSDL} Left channel, Morlet
cycle-octave scalogram envelope, medium time resolution, spectral decay. }
\end{center}
\end{figure}

\begin{figure}
\begin{center}
\includegraphics[width=\pctwidth\textwidth,keepaspectratio]{figures/SR-R-MorletScalogramMTRENVSD}
\caption{\label{MorletScalogramMTRENVSDR} Right channel, Morlet
cycle-octave scalogram envelope, medium time resolution, spectral decay. }
\end{center}
\end{figure}

\begin{figure}
\begin{center}
\includegraphics[width=\pctwidth\textwidth,keepaspectratio]{figures/SR-L-MorletScalogramMTRENVSF}
\caption{\label{MorletScalogramMTRENVSFL} Left channel, Morlet
cycle-octave scalogram envelope, medium time resolution, spectral formation. }
\end{center}
\end{figure}

\begin{figure}
\begin{center}
\includegraphics[width=\pctwidth\textwidth,keepaspectratio]{figures/SR-R-MorletScalogramMTRENVSF}
\caption{\label{MorletScalogramMTRENVSFR} Right channel, Morlet
cycle-octave scalogram envelope, medium time resolution, spectral formation. }
\end{center}
\end{figure}

\begin{figure}
\begin{center}
\includegraphics[width=\pctwidth\textwidth,keepaspectratio]{figures/SR-L-MorletScalogramMTRENVMap}
\caption{\label{MorletScalogramMTRENVMapL}   Left channel, Morlet
cycle-octave scalogram envelope, medium time resolution, colored map. }
\end{center}
\end{figure}

\begin{figure}
\begin{center}
\includegraphics[width=\pctwidth\textwidth,keepaspectratio]{figures/SR-R-MorletScalogramMTRENVMap}
\caption{\label{MorletScalogramMTRENVMapR}   Right channel, Morlet
cycle-octave scalogram envelope, medium time resolution, colored map. }
\end{center}
\end{figure}

\begin{figure}
\begin{center}
\includegraphics[width=\pctwidth\textwidth,keepaspectratio]{figures/SR-L-MorletScalogramMTRETCSD}
\caption{\label{MorletScalogramMTRETCSDL} Left channel, Morlet
cycle-octave scalogram ETC, medium time resolution, spectral decay. }
\end{center}
\end{figure}

\begin{figure}
\begin{center}
\includegraphics[width=\pctwidth\textwidth,keepaspectratio]{figures/SR-R-MorletScalogramMTRETCSD}
\caption{\label{MorletScalogramMTRETCSDR} Right channel, Morlet
cycle-octave scalogram ETC, medium time resolution, spectral decay. }
\end{center}
\end{figure}

\begin{figure}
\begin{center}
\includegraphics[width=\pctwidth\textwidth,keepaspectratio]{figures/SR-L-MorletScalogramMTRETCSF}
\caption{\label{MorletScalogramMTRETCSFL} Left channel, Morlet
cycle-octave scalogram ETC, medium time resolution, spectral formation. }
\end{center}
\end{figure}

\begin{figure}
\begin{center}
\includegraphics[width=\pctwidth\textwidth,keepaspectratio]{figures/SR-R-MorletScalogramMTRETCSF}
\caption{\label{MorletScalogramMTRETCSFR} Right channel, Morlet
cycle-octave scalogram ETC, medium time resolution, spectral formation. }
\end{center}
\end{figure}

\begin{figure}
\begin{center}
\includegraphics[width=\pctwidth\textwidth,keepaspectratio]{figures/SR-L-MorletScalogramMTRETCMap}
\caption{\label{MorletScalogramMTRETCMapL} Left channel, Morlet
cycle-octave scalogram ETC, medium time resolution, colored map. }
\end{center}
\end{figure}

\begin{figure}
\begin{center}
\includegraphics[width=\pctwidth\textwidth,keepaspectratio]{figures/SR-R-MorletScalogramMTRETCMap}
\caption{\label{MorletScalogramMTRETCMapR} Right  channel, Morlet
cycle-octave scalogram ETC, medium time resolution, colored map. }
\end{center}
\end{figure}

\clearpage
\subsection{Baseline}
\label{SampleResultsBaseline}

The following series of graphs show the comparison between a Dirac delta
and the  corrected left  channel.  The Dirac  delta is the  mathematical
representation  of a ``perfect''  system  i.e. a system  which outputs a
perfect copy of its input. Looking at these graphs it is possible to see
both what is left uncorrected by DRC and what a ``perfect'' system looks
like on this kind of graphs. The  graphs are presented in the same order
of the previous graphs.

\subsubsection{Baseline time response}
\label{BaselineTimeResponse}
\clearpage

\begin{figure}
\begin{center}
\includegraphics[width=\pctwidth\textwidth,keepaspectratio]{figures/SR-B-IRStepResponse}
\caption{\label{BaselineStepResponseFullRange} Step response}
\includegraphics[width=\pctwidth\textwidth,keepaspectratio]{figures/SR-B-IRFullRange}
\caption{\label{BaselineImpulseResponseFullRange} Full range impulse response.}
\end{center}
\end{figure}

\begin{figure}
\begin{center}
\includegraphics[width=\pctwidth\textwidth,keepaspectratio]{figures/SR-B-IRFullRangeEnvelope}
\caption{\label{BaselineEnvelopeFullRange}  Full  range impulse response
envelope.}
\includegraphics[width=\pctwidth\textwidth,keepaspectratio]{figures/SR-B-IRFullRangeETC}
\caption{\label{BaselineETCFullRange}Full range time-energy response.}
\end{center}
\end{figure}

\begin{figure}
\begin{center}
\includegraphics[width=\pctwidth\textwidth,keepaspectratio]{figures/SR-B-IR2kHzBrickwall}
\caption{\label{BaselineImpulseResponse2kHzBrickwall}   Impulse response
after brickwall filtering at 2 kHz.}
\includegraphics[width=\pctwidth\textwidth,keepaspectratio]{figures/SR-B-IR2kHzBrickwallEnvelope}
\caption{\label{BaselineEnvelope2kHzBrickwall}   Impulse  response envelope after
brickwall filtering at 2 kHz.}
\end{center}
\end{figure}

\begin{figure}
\begin{center}
\includegraphics[width=\pctwidth\textwidth,keepaspectratio]{figures/SR-B-IR2kHzBrickwallETC}
\caption{\label{BaselineETC2kHzBrickwall}    Time-energy  response after
brickwall filtering at 2 kHz.}
\includegraphics[width=\pctwidth\textwidth,keepaspectratio]{figures/SR-B-IR200HzBrickwall}
\caption{\label{BaselineImpulseResponse200HzBrickwall}  Impulse response
after brickwall filtering at 200 Hz.}
\end{center}
\end{figure}

\begin{figure}
\begin{center}
\includegraphics[width=\pctwidth\textwidth,keepaspectratio]{figures/SR-B-IR200HzBrickwallEnvelope}
\caption{\label{BaselineEnvelope200HzBrickwall}        Impulse  response
envelope after brickwall filtering at 200 Hz.}
\includegraphics[width=\pctwidth\textwidth,keepaspectratio]{figures/SR-B-IR200HzBrickwallETC}
\caption{\label{BaselineETC200HzBrickwall}  Time-energy response after brickwall
filtering at 200 Hz.}
\end{center}
\end{figure}

\clearpage
\subsubsection{Baseline frequency response}
\label{BaselineFrequencyResponse}
\clearpage

\begin{figure}
\begin{center}
\includegraphics[width=\pctwidth\textwidth,keepaspectratio]{figures/SR-B-MRUnsmoothed1ms}
\caption{\label{BaselineMagnitudeUnsmoothed1ms}    Unsmoothed  frequency
response magnitude, 1 ms Blackman window.}
\includegraphics[width=\pctwidth\textwidth,keepaspectratio]{figures/SR-B-MRUnsmoothed5ms}
\caption{\label{BaselineMagnitudeUnsmoothed5ms}    Unsmoothed  frequency
response magnitude, 5 ms Blackman window.}
\end{center}
\end{figure}

\begin{figure}
\begin{center}
\includegraphics[width=\pctwidth\textwidth,keepaspectratio]{figures/SR-B-MRUnsmoothed200ms}
\caption{\label{BaselineMagnitudeUnsmoothed200ms}   Unsmoothed frequency
response magnitude, bass range, 200 ms Blackman window.}
\includegraphics[width=\pctwidth\textwidth,keepaspectratio]{figures/SR-B-MRFDWSmoothed}
\caption{\label{BaselineFDWSmoothedNormal}  Frequency response magnitude
smoothed using a frequency  dependent windowing  with windowing settings
close to those used by the normal.drc sample configuration file.}
\end{center}
\end{figure}

\begin{figure}
\begin{center}
\includegraphics[width=\pctwidth\textwidth,keepaspectratio]{figures/SR-B-MRFDWSmoothed-1-6}
\caption{\label{BaselineMagnitudeFDWSmoothed16Oct}    Frequency response
magnitude  smoothed using a  frequency  dependent windowing  providing a
frequency resolution close to 1/6 of octave smoothing.}
\includegraphics[width=\pctwidth\textwidth,keepaspectratio]{figures/SR-B-MRFDWSmoothed-1-3}
\caption{\label{BaselineMagnitudeFDWSmoothed13Oct}    Frequency response
magnitude  smoothed using a  frequency  dependent windowing  providing a
frequency resolution close to 1/3 of octave smoothing.}
\end{center}
\end{figure}

\begin{figure}
\begin{center}
\includegraphics[width=\pctwidth\textwidth,keepaspectratio]{figures/SR-B-MRBarkSmoothed}
\caption{\label{BaselineMagnitudeBarkSmoothed} Frequency response
magnitude smoothed over the Bark psychoacoustic scale with many different Blackman windows applied.}
\includegraphics[width=\pctwidth\textwidth,keepaspectratio]{figures/SR-B-MRERBSmoothed}
\caption{\label{BaselineMagnitudeERBSmoothed} Frequency response
magnitude smoothed over the Bark psychoacoustic scale with many different Blackman windows applied.}
\end{center}
\end{figure}

\clearpage
\subsubsection{Baseline phase response}
\label{BaselinePhaseResponse}
\clearpage

\begin{figure}
\begin{center}
\includegraphics[width=\pctwidth\textwidth,keepaspectratio]{figures/SR-B-PRUnsmoothed1ms}
\caption{\label{BaselinePhaseResponse1}  Unsmoothed phase response, 1 ms
Blackman window.}
\includegraphics[width=\pctwidth\textwidth,keepaspectratio]{figures/SR-B-PRUnsmoothed5ms}
\caption{\label{BaselinePhaseResponse5}  Unsmoothed phase response, 5 ms
Blackman window.}
\end{center}
\end{figure}

\begin{figure}
\begin{center}
\includegraphics[width=\pctwidth\textwidth,keepaspectratio]{figures/SR-B-PRUnsmoothed200ms}
\caption{\label{BaselinePhaseResponse200} Unsmoothed phase response, 200
ms Blackman window.}
\includegraphics[width=\pctwidth\textwidth,keepaspectratio]{figures/SR-B-PRFDWSmoothed}
\caption{\label{BaselinePhaseFDWSmoothedNormal}  Phase response smoothed
using a frequency dependent  windowing with  windowing settings close to
those used by the normal.drc sample configuration file.}
\end{center}
\end{figure}

\begin{figure}
\begin{center}
\includegraphics[width=\pctwidth\textwidth,keepaspectratio]{figures/SR-B-PRFDWSmoothed-1-6}
\caption{\label{BaselinePhaseFDWSmoothed16Oct}  Phase response magnitude
smoothed using  a frequency  dependent  windowing providing  a frequency
resolution close to 1/6 of octave smoothing.}
\includegraphics[width=\pctwidth\textwidth,keepaspectratio]{figures/SR-B-PRFDWSmoothed-1-3}
\caption{\label{BaselinePhaseFDWSmoothed13Oct}  Phase response magnitude
smoothed using  a frequency  dependent  windowing providing  a frequency
resolution close to 1/3 of octave smoothing.}

\end{center}
\end{figure}

\clearpage
\subsubsection{Baseline time-frequency analysis}
\label{BaselineTimeFrequencyAnalysis}
\clearpage

\begin{figure}
\begin{center}
\includegraphics[width=\pctwidth\textwidth,keepaspectratio]{figures/SR-B-SDHighRange}
\caption{\label{BaselineSpectralDecayHR}  Left  channel, spectral decay,
high range. Spectral  decay from 2  kHz to 20 kHz with  a 0.5 ms sliding
Blackman window.}
\end{center}
\end{figure}

\begin{figure}
\begin{center}
\includegraphics[width=\pctwidth\textwidth,keepaspectratio]{figures/SR-B-SFHighRange}
\caption{\label{BaselineSpectralFormationHR}     Left channel,  spectral
formation, high  range. Spectral  formation from 2 kHz  to 20 kHz with a
0.5 ms sliding Blackman window.}
\end{center}
\end{figure}

\begin{figure}
\begin{center}
\includegraphics[width=\pctwidth\textwidth,keepaspectratio]{figures/SR-B-SDMidRange}
\caption{\label{BaselineSpectralDecayMR}  Left channel, spectral decay,
mid range.  Spectral decay  from 200 Hz to  2000 Hz with  a 5 ms sliding
Blackman window.}
\end{center}
\end{figure}

\begin{figure}
\begin{center}
\includegraphics[width=\pctwidth\textwidth,keepaspectratio]{figures/SR-B-SFMidRange}
\caption{\label{BaselineSpectralFormationMR}     Left channel,  spectral
formation, mid range. Spectral formation from 200 Hz to 2000 Hz with a 5
ms sliding Blackman window.}
\end{center}
\end{figure}

\begin{figure}
\begin{center}
\includegraphics[width=\pctwidth\textwidth,keepaspectratio]{figures/SR-B-SDBassRange}
\caption{\label{BaselineSpectralDecayBR}  Left  channel, spectral decay,
bass range.  Spectral decay  from 20 Hz to  200 Hz with a  50 ms sliding
Blackman window.}
\end{center}
\end{figure}

\begin{figure}
\begin{center}
\includegraphics[width=\pctwidth\textwidth,keepaspectratio]{figures/SR-B-SFBassRange}
\caption{\label{BaselineSpectralFormationBR}    Left  channel, spectral
formation, bass range. Spectral formation from 20 Hz to 200 Hz with a 50
ms sliding Blackman window.}
\end{center}
\end{figure}

\begin{figure}
\begin{center}
\includegraphics[width=\pctwidth\textwidth,keepaspectratio]{figures/SR-B-SDHighRangeW}
\caption{\label{BaselineSpectralDecayHRW}  Left  channel, spectral decay,
high range. Spectral  decay from 2  kHz to 20 kHz with  a 1.0 ms sliding
Blackman window.}
\end{center}
\end{figure}

\begin{figure}
\begin{center}
\includegraphics[width=\pctwidth\textwidth,keepaspectratio]{figures/SR-B-SFHighRangeW}
\caption{\label{BaselineSpectralFormationHRW}     Left channel,  spectral
formation, high  range. Spectral  formation from 2 kHz  to 20 kHz with a
1.0 ms sliding Blackman window.}
\end{center}
\end{figure}

\begin{figure}
\begin{center}
\includegraphics[width=\pctwidth\textwidth,keepaspectratio]{figures/SR-B-SDMidRangeW}
\caption{\label{BaselineSpectralDecayMRW}  Left channel, spectral decay,
mid range.  Spectral decay  from 200 Hz to  2000 Hz with  a 10 ms sliding
Blackman window.}
\end{center}
\end{figure}

\begin{figure}
\begin{center}
\includegraphics[width=\pctwidth\textwidth,keepaspectratio]{figures/SR-B-SFMidRangeW}
\caption{\label{BaselineSpectralFormationMRW}     Left channel,  spectral
formation, mid range. Spectral formation from 200 Hz to 2000 Hz with a 10
ms sliding Blackman window.}
\end{center}
\end{figure}

\begin{figure}
\begin{center}
\includegraphics[width=\pctwidth\textwidth,keepaspectratio]{figures/SR-B-SDBassRangeW}
\caption{\label{BaselineSpectralDecayBRW}  Left  channel, spectral decay,
bass range.  Spectral decay  from 20 Hz to  200 Hz with a  50 ms sliding
Blackman window.}
\end{center}
\end{figure}

\begin{figure}
\begin{center}
\includegraphics[width=\pctwidth\textwidth,keepaspectratio]{figures/SR-B-SFBassRangeW}
\caption{\label{BaselineSpectralFormationBRW}    Left  channel, spectral
formation, bass range. Spectral formation from 20 Hz to 200 Hz with a 100
ms sliding Blackman window.}
\end{center}
\end{figure}

\begin{figure}
\begin{center}
\includegraphics[width=\pctwidth\textwidth,keepaspectratio]{figures/SR-B-Spectrogram01ms}
\caption{\label{BaselineHigResSpectrogram01ms}         High   resolution
spectrograms  from -10 ms to  40 ms, 1 ms  Blackman window,  90 dB level
range, left channel.  The frequency  range is from DC  to Nyquist (22050
Hz) on a linear scale.}
\end{center}
\end{figure}

\begin{figure}
\begin{center}
\includegraphics[width=\pctwidth\textwidth,keepaspectratio]{figures/SR-B-Spectrogram20ms}
\caption{\label{BaselineHigResSpectrogram20ms}         High   resolution
spectrograms from -100 ms to 400 ms,  20 ms Blackman window, 90 dB level
range, left channel.  The frequency  range is from DC  to Nyquist (22050
Hz) on a linear scale.}
\end{center}
\end{figure}

\clearpage
\subsubsection{Baseline wavelet cycle-octave analysis}
\label{BaselineWaveletAnalysis}

% Morlet cycle-octave scalogram, alta risoluzione

\clearpage
\begin{figure}
\begin{center}
\includegraphics[width=\pctwidth\textwidth,keepaspectratio]{figures/SR-B-MorletScalogramHTRENVSD}
\caption{\label{MorletScalogramHTRENVSDB} Baseline, Morlet
cycle-octave scalogram envelope, high time resolution, spectral decay. }
\end{center}
\end{figure}

\begin{figure}
\begin{center}
\includegraphics[width=\pctwidth\textwidth,keepaspectratio]{figures/SR-B-MorletScalogramHTRENVSF}
\caption{\label{MorletScalogramHTRENVSFB} Baseline, Morlet
cycle-octave scalogram envelope, high time resolution, spectral formation. }
\end{center}
\end{figure}

\begin{figure}
\begin{center}
\includegraphics[width=\pctwidth\textwidth,keepaspectratio]{figures/SR-B-MorletScalogramHTRENVMap}
\caption{\label{MorletScalogramHTRENVMapB}   Baseline, Morlet
cycle-octave scalogram envelope, high time resolution, colored map. }
\end{center}
\end{figure}

\begin{figure}
\begin{center}
\includegraphics[width=\pctwidth\textwidth,keepaspectratio]{figures/SR-B-MorletScalogramHTRETCSD}
\caption{\label{MorletScalogramHTRETCSDB} Baseline, Morlet
cycle-octave scalogram ETC, high time resolution, spectral decay. }
\end{center}
\end{figure}

\begin{figure}
\begin{center}
\includegraphics[width=\pctwidth\textwidth,keepaspectratio]{figures/SR-B-MorletScalogramHTRETCSF}
\caption{\label{MorletScalogramHTRETCSFB} Baseline, Morlet
cycle-octave scalogram ETC, high time resolution, spectral formation. }
\end{center}
\end{figure}

\begin{figure}
\begin{center}
\includegraphics[width=\pctwidth\textwidth,keepaspectratio]{figures/SR-B-MorletScalogramHTRETCMap}
\caption{\label{MorletScalogramHTRETCMapB} Baseline, Morlet
cycle-octave scalogram ETC, high time resolution, colored map. }
\end{center}
\end{figure}

% Morlet cycle-octave scalogram, media risoluzione

\clearpage
\begin{figure}
\begin{center}
\includegraphics[width=\pctwidth\textwidth,keepaspectratio]{figures/SR-B-MorletScalogramMTRENVSD}
\caption{\label{MorletScalogramMTRENVSDB} Baseline, Morlet
cycle-octave scalogram envelope, medium time resolution, spectral decay. }
\end{center}
\end{figure}

\begin{figure}
\begin{center}
\includegraphics[width=\pctwidth\textwidth,keepaspectratio]{figures/SR-B-MorletScalogramMTRENVSF}
\caption{\label{MorletScalogramMTRENVSFB} Baseline, Morlet
cycle-octave scalogram envelope, medium time resolution, spectral formation. }
\end{center}
\end{figure}

\begin{figure}
\begin{center}
\includegraphics[width=\pctwidth\textwidth,keepaspectratio]{figures/SR-B-MorletScalogramMTRENVMap}
\caption{\label{MorletScalogramMTRENVMapB}   Baseline, Morlet
cycle-octave scalogram envelope, medium time resolution, colored map. }
\end{center}
\end{figure}

\begin{figure}
\begin{center}
\includegraphics[width=\pctwidth\textwidth,keepaspectratio]{figures/SR-B-MorletScalogramMTRETCSD}
\caption{\label{MorletScalogramMTRETCSDB} Baseline, Morlet
cycle-octave scalogram ETC, medium time resolution, spectral decay. }
\end{center}
\end{figure}

\begin{figure}
\begin{center}
\includegraphics[width=\pctwidth\textwidth,keepaspectratio]{figures/SR-B-MorletScalogramMTRETCSF}
\caption{\label{MorletScalogramMTRETCSFB} Baseline, Morlet
cycle-octave scalogram ETC, medium time resolution, spectral formation. }
\end{center}
\end{figure}

\begin{figure}
\begin{center}
\includegraphics[width=\pctwidth\textwidth,keepaspectratio]{figures/SR-B-MorletScalogramMTRETCMap}
\caption{\label{MorletScalogramMTRETCMapB} Baseline, Morlet
cycle-octave scalogram ETC, medium time resolution, colored map. }
\end{center}
\end{figure}

\end{document}
